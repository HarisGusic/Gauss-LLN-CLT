
\indent
U ovom radu su detaljno prezentovane tri teme koje su izuzetno važne u oblasti
vjerovatnoće i statistike.  \\

\textit{Normalna raspodjela} je jedna od najčesće susretanih raspodjela vjerovatnoće u
jako velikom broju naučnih i inženjerskih oblasti, kao i u svakodnevnom životu.
U ovom radu su prezentovane normalne raspodjele za slučaj jedne, kao i za slučaj
više varijabli, uz detaljnu ilustraciju ovih funkcija za neke karakteristične
vrijednosti parametara. Ova raspodjela ima neke vrlo korisne osobine koje će
biti predstavljene u nastavku rada. Neke od tih osobina su direktno povezane sa
centralnim graničnim teoremom.

\textit{Zakon velikih brojeva} je značajan rezultat koji opravdava neke
intuicije vezane za sam pojam vjerovatnoće. Jedna takva intuicija jeste da će
relativna frekvencija nekog događaja u nizu eksperimenata biti približno jednaka
vjerovatnoći tog događaja. Druga je da će srednja vrijednost slučajne varijable
dobivena eksperimentom biti približno jednaka očekivanoj vrijednosti dobivenoj
na osnovu probabilističkog modela. Alternativno tumačenje je da zakon velikih
brojeva opravdava primjenjivost aksiomatske teorije vjerovatnoće u praksi. U
nastavku rada su navedene još neke korisne implikacije. Zakon velikih brojeva je
pogodno opisati korištenjem pojma slučajne sekvence (procesa), što će biti
urađeno u nastavku rada.

\textit{Centralni granični teorem} daje objašnjenje zašto se normalna raspodjela
susreće toliko često u svakodnevnom životu i u praksi. Ovaj teorem objašnjava
ponašanje vjerovatnoće sume velikog broja slučajnih varijabli, pod određenim
uslovima koji su dosta blagi sa aspekta praktičnih primjena. U najjednostavnijem
obliku teorem govori da će suma nezavisnih slučajnih varijabli imati približno
normalnu raspodjelu.  Zavisno od karaktera slučajnih varijabli, konvergencija ka
normalnoj raspodjeli može biti izuzetno brza. Fokus ovog razmatranja je na
jednostavnijim primjerima, i to za slučaj jedne varijable. Radi potpunosti su
navedene i neke generalnije formulacije teorema. \\
