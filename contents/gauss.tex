
U prirodi se najčešće susreće normalna raspodjela, koja je opisana funkcijom
gustine raspodjele vjerovatnoće:

$$p_X(x) = \mathcal{N}(\mu,\sigma^2)
	= \frac{1}{\sqrt{2\pi\sigma^2}} e^{-\frac{(x-\mu)^2}{2\sigma^2}}$$

Slučajna varijabla je normalno raspodijeljena ako ima funkciju gustine
raspodjele vjerovatnoće (pdf) oblika:

\subsection{Multivarijabilna normalna raspodjela}

Normalna raspodjela se može generalizirati i za vektore slučajnih varijabli.
Za slučajni vektor $\bm X \in \mathbb{R}^k$ se kaže da je normalno
raspodijeljen ako je njegova funkcija gustine raspodjele:

$$p_{\bm X}(\bm x) = \mathcal{N}(\bm\mu,\bm C_{\bm X\bm X}) =
	\frac{1}{\sqrt{(2\pi)^k \det \bm C_{\bm X \bm X}}}
	e^{
		-\frac{1}{2}(\bm x - \bm\mu_{\bm X})^\text T
		\bm C_{\bm X \bm X}^{-1}
		(\bm x - \bm\mu_{\bm X})
 	 }$$

Jedini uslov koji se postavlja na parametre ove raspodjele jeste da matrica
$\bm C_{\bm X\bm X}$ bude ispravna kovarijantna matrica, tj. da bude simetrična
i pozitivno semi-definitna.

\begin{theorem}
	\label{th:lin-tr-gauss}
	Neka je $\bm X \in \mathbb{R}^n$ normalno raspodijeljen slučajni vektor sa
	srednjom vrijednosti $\bm\mu_{\bm X}$ i kovarijantnom matricom $\bm C_{\bm
	X\bm X}$. Neka je $\bm A \in \mathbb{R}^{m\times n}$, $\bm b \in
	\mathbb{R}^n$, pri čemu je $m\le n$. Tada je vektor $\bm Y = \bm A \bm X + \bm
	b$ također normalno raspodijeljen sa srednjom vrijednosti $\bm\mu_{\bm Y} =
	\bm A\bm\mu_{\bm X} + \bm b$ i kovarijantnom matricom $\bm C_{\bm Y\bm Y} =
	\bm A \bm C_{\bm X\bm X} \bm A^\mathrm T$.

\end{theorem}

Jednostavnije rečeno, svaka linearna transformacija Gaussovog vektora uz
eventualnu translaciju, ponovo daje Gaussov vektor.

\begin{corollary}
	Svaki podvektor normalno raspodijeljenog slučajnog vektora $\bm X$ je također
	normalno raspodijeljen. Drugim riječima, sve marginalne pdf slučajnog
	vektora $\bm X$ su također normalne raspodjele.
\end{corollary}

Ovo je lako pokazati. Naime, neki podvektor $\bm Y = [X_{i_1}\ X_{i_2}\
	 \cdots\ X_{i_m}]^\text T$ vektora $\bm X$ se može dobiti formiranjem matrice
$\bm A$ koja u $j$ - tom redu sadrži jedinicu na poziciji $i_j$, dok su
ostali elementi jednaki nuli. Za vektor $\bm b$ se uzima nul-vektor.
Parametri raspodjele novodobijenog vektora se jednostavno određuju primjenom
teorema \ref{th:lin-tr-gauss}.

\begin{exmp}
	Neka je data normalna raspodjela $p_{\bm X}(\bm x) = p_{X_1,X_2}(x_1, x_2) =
	\mathcal{N}(\bm\mu, \bm \Sigma)$. Odrediti marginalne pdf $p_{X_1}(x)$ i
	$p_{X_2}(x)$.
\end{exmp}

Neka je:

\begin{eqnarray}
	\bm\mu = \left[\begin{array}{c}
		\mu_1 \\ \mu_2
	\end{array}\right],\ 
	\bm\Sigma = \left[\begin{array}{cc}
	  \sigma_1^2 & \sigma_{12} \\ \sigma_{12} & \sigma_2^2
	\end{array}\right]
\end{eqnarray}

Slučajne varijable $X_1$ i $X_2$ se mogu zapisati na sljedeći način:
\newcommand*{\vecrow}[2]{\left[\begin{array}{cc}#1&#2\end{array}\right]}
\newcommand*{\veccol}[2]{\left[\begin{array}{c}#1\\#2\end{array}\right]}
\begin{eqnarray}
	X_1 = \vecrow{1}{0} \veccol{X_1}{X_2} = \bm A_1 \bm X,
	\\
	X_2 = \vecrow{0}{1} \veccol{X_1}{X_2} = \bm A_2 \bm X,
\end{eqnarray}

Primjenom teoreme \ref{th:lin-tr-gauss} se zaključuje da će $p_{X_1}$ i
$p_{X_2}$ također biti normalne raspodjele:

\begin{align*}
	p_{X_1} &= \mathcal{N}(\bm A_1\bm\mu, \bm A_1 \bm\Sigma \bm A_1^\mathrm T)
		= \mathcal{N}(\mu_1, \sigma_1^2) \\
	p_{X_2} &= \mathcal{N}(\bm A_2\bm\mu, \bm A_2 \bm\Sigma \bm A_2^\mathrm T)
		= \mathcal{N}(\mu_2, \sigma_2^2)
\end{align*}

