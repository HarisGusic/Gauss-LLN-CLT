U ovom poglavlju će biti objašnjen jedan od najznačajnijih rezultata u oblasti
vjerovatnoće. Centralni granični teorem opisuje ponašanje sume velikog broja
slučajnih varijabli. Prije nego što se formuliše teorem, biće napravljeno uvodno
razmatranje.

\subsection{Raspodjela sume slučajnih varijabli}

Neka su $K$ i $L$ dvije diskretne slučajne varijable sa masenim funkcijama
raspodjele (pmf) $p_K(k)$ i $p_L(l)$ respektivno. Neka je potrebno naći pmf
$p_M(m)$ slučajne varijable $M=K+L$.
%
\newcommand{\pr}{\text{Pr}}
\begin{align*}
  p_M(m) = \pr(M=m) = \pr(k\in\mathbb{Z}, l=m-k)
  = \sum_{k=-\infty}^{\infty} p_{K,L}(k,m-k)
\end{align*}

Ako su varijable $K$ i $L$ međusobno nezavisne, onda vrijedi $p_{K,L}(k,l) =
p_K(k)p_L(l)$, te se prethodni izraz može dodatno pojednostaviti:
%
\begin{align*}
  p_M(m) = \sum_{k=-\infty}^{\infty} p_K(k)p_L(m-k) = p_K * p_L (m)
\end{align*}

Dakle, suma nezavisnih diskretnih slučajnih varijabli će imati pmf koja je
jednaka diskretnoj konvoluciji pmf tih varijabli. Sličan zaključak vrijedi i za
slučaj kontinualnih varijabli.

\begin{theorem} %TODO da li ga nazvati teoremom?

  Ako su $X_1,X_2,...,X_n$ međusobno nezavisne slučajne varijable sa funkcijama
  gustine vjerovatnoće $p_{X_i}$, $i=1,...,n$ respektivno, onda njihova suma $S$
  ima funkciju gustine vjerovatnoće:
  $$p_S(s) = p_{X_1} * p_{X_2} * \cdots * p_{X_n} (s)$$

\end{theorem}

Prethodno razmatranje će sada biti intuitivno objašnjeno na primjeru.  Radi
jednostavnosti, razmatranje će biti ograničeno na diskretni slučaj.

\begin{exmp} \label{ex:novcic}
  
Neka su $X_i$ slučajne varijable koje uzimaju vrijednosti iz $\{0,1\}$ zavisno
od ishoda bacanja nepravednog novčića, pri čemu je $p_X(0)= p$, $p_X(1)=q=1-p$.
Potrebno je odrediti broj bacanja $K$ u kojima je $X_i=1$, ako je novčić bačen
$n$ puta.

\end{exmp}

Traženi broj $K$ također predstavlja slučajnu varijablu i može se izraziti
kao:\footnote{Pošto $X_i$ mogu uzeti samo vrijednosti 0 ili 1, njihova suma je
jednaka broju njih koji su jednaki 1}

$$K = \sum_{i=1}^{n} X_i$$
\\

Pošto su slučajne varijable $X_i$ IID, pmf od $K$ se dobija konvolucijom:
$$p_K(k) = \underbrace{p_X * p_X * \cdots * p_X}_{n\text{ puta}}(k)$$

Jednostavan način da se odredi ova konvolucija jeste korištenjem
$\mathcal{Z}$ transformacije.\footnote{\ $p_K(k)$ se može odrediti i čisto
kombinatornim pristupom.}
Naime, vrijedi: %TODO footnote binomna
\begin{equation}
  \mathcal{Z}\{p_K(k)\}
  = \left[\mathcal{Z}\{p_X(k)\}\right]^n
  = (p+qz^{-1})^n
  = \sum_{k=0}^{n} \left(\frac{n}{k}\right) p^k q^{n-k} z^{-k}
\end{equation}

Direktnom primjenom inverzne $ \mathcal{Z}$ transformacije se dobiva:
$$p_K(k) = \left(\frac{n}{k}\right) p^kq^{n-k}$$

\begin{equation} \label{eq:binom-pmf}
  p_K(k) = \frac{1}{2^n} \left(\frac{n}{k}\right)
\end{equation}

Na slikama \ref{fig:binom} je prikazan izgled pmf iz \eqref{eq:binom-pmf} za
razne vrijednosti $n$.

Na slici \ref{fig:binom:a} je prikazana pmf $p_K(k)$, za jedno bacanje, koja se
naravno podudara sa $p_X(k)$. Postoje dva moguća ishoda, sa jednakom
vjerovatnoćom koja iznosi 0.5. Ako se novčić baci drugi put (slika
\ref{fig:binom:b}), mogući ishodi za $K$ su $0, 1, 2$, pri čemu je ishod $1$
dvostruko vjerovatniji od ostalih, jer se može dobiti na dva načina
($1=0+1=1+0$), dok se ostali mogu dobiti na samo jedan način ($0=0+0$, $2=1+1$).
Ako se novčić baci treći put (slika \ref{fig:binom:c}), mogući ishodi su
$0,1,2,3$, pri čemu vrijednosti 1 i 2 imaju najveću vjerovatnoću. Na slikama
\ref{fig:binom:d} i \ref{fig:binom:e} su prikazane pmf za $n=8$ i $n=50$
respektivno. Već na slici \ref{fig:binom:d} se može primijetiti da je pmf
skoncentrisana oko sredine intervala, dok se na slici \ref{fig:binom:e} vidi da
pmf aproksimira Gauss-ovu funkciju. Ova pojava se može formalno iskazati
\textit{De Moivre-Laplace}-ovim teoremom koji tvrdi da je:
\begin{equation}
  \left(\frac{n}{k}\right) p^kq^{n-k} \simeq \frac{1}{\sqrt{2\pi npq}}
  e^{-\frac{(k-np)^2}{2npq}}
\end{equation}
Dakle, binomna raspodjela aproksimira Gauss-ovu funkciju za veliko $n$.
Prethodni primjer je dosta regularan zbog činjenice da je novčić pravedan, pa su
sve pmf simetrične oko neke centralne tačke. U nastavku su navedeni neki bitni
zaključci, koji vrijede u općem slučaju. Ovi zaključci direktno slijede iz
osobina konvolucije, a također su i dosta intuitivni.
TODO
%\begin{enumerate}
  %\item Suma dvije varijable uzima 
%\end{enumerate}

\begin{figure}[H]
  \centering
  \begin{subfigure}[b]{0.3\textwidth}
    \centering
    \includegraphics[width=\textwidth]{clt_binom_1.pdf}
    \caption{$n=1$}
    \label{fig:binom:a}
  \end{subfigure}
  \vspace{10pt}
	\begin{subfigure}[b]{0.3\textwidth}
		\centering
    \includegraphics[width=\textwidth]{clt_binom_2.pdf}
    \caption{$n=2$}
    \label{fig:binom:b}
	\end{subfigure}
	\begin{subfigure}[b]{0.3\textwidth}
		\centering
    \includegraphics[width=\textwidth]{clt_binom_3.pdf}
    \caption{$n=3$}
    \label{fig:binom:c}
	\end{subfigure}
	\begin{subfigure}[b]{0.3\textwidth}
		\centering
    \includegraphics[width=\textwidth]{clt_binom_8.pdf}
    \caption{$n=8$}
    \label{fig:binom:d}
	\end{subfigure}
	\begin{subfigure}[b]{0.3\textwidth}
		\centering
    \includegraphics[width=\textwidth]{clt_binom_50.pdf}
    \caption{$n=50$}
    \label{fig:binom:e}
	\end{subfigure}
	\caption{Binomna raspodjela i De Moivre-Laplace aproksimacija}
  \label{fig:binom}
\end{figure}

\subsection{Formulacija centralnog graničnog teorema}

Postoje različite formulacije centralnog graničnog teorema (engl.
\textit{central limit theorem} - CLT), ovdje će biti data jedna od
najjednostavnijih, mada ne i najopćenitija.
% TODO u uvodu dati prevod CLT
\begin{theorem} \label{th:clt}
  Neka je $\{X_i, i=1,2,...\}$ sekvenca IID slučajnih varijabli sa
  konačnim očekivanjem $\mu$ i konačnom varijansom $\sigma^2$. Tada
  sekvenca
  \begin{equation}
    S_n = \frac{1}{\sqrt{n}} \sum_{i=1}^{n} \frac{X_i-\mu}{\sigma}
  \end{equation}
  konvergira po distribuciji ka normalnoj raspodjeli $\mathcal{N}(0, 1)$.
\end{theorem}

Nažalost, način kako je definirana sekvenca $S_n$ nije najbolji oblik za
intuitivno shvatanje teorema. Prirodnije bi bilo sekvencu $S_n$ definirati kao
$S_n=\sum_{i=1}^{n} X_i$. Međutim, ukoliko se koristi takva definicija, nastaje
problem u tome što takva sekvenca neće konvergirati po distribuciji. Raspodjela
takve sekvence će sa povećanjem $n$ bolje i bolje aproksimirati \textbf{neku}
Gauss-ovu raspodjelu $\mathcal{N}(n\mu,n\sigma^2)$, ali očigledno ovakva sekvenca
raspodjela neće težiti ka istoj raspodjeli.

\setcounter{corollary}{0}
\begin{corollary}

  Suma velikog broja IID slučajnih varijabli sa očekivanjem $\mu$ i varijansom
  $\sigma^2$ imati raspodjelu koja aproksimira $\mathcal{N}(n\mu, n\sigma^2)$,
  nezavisno od raspodjele varijabli koje učestvuju u sumi. 

\end{corollary}

Ovakva interpretacija teorema se manje odupire intuiciji nego sama formulacija
teorema.

%TODO ref to eqn

%TODO teorem ne pomaže intuiciji

\begin{corollary}
  Konvolucijom pozitivne funkcije sa samom sobom mnogo puta dobiva se približno
  Gauss-ova funkcija.
\end{corollary}

Ovo je već ilustrovano u primjeru \ref{ex:novcic}, mada je tamo bila razmatrana
raspodjela u diskretnom slučaju koja je dosta regularna i simetrična. U nastavku
će isti princip biti prikazan na nekoliko različitih kontinualnih funkcija. Neka
je $f(x)$ neka nenegativna funkcija, i neka je:

\begin{equation}
  g_n(x) = f^{*n}(x) := \underbrace{f*f\cdots*f}_{n\text{ puta}}\ (x)
\end{equation}

Također, neka je $\widetilde{g}_n(x)$ aproksimacija funkcije $g_n(x)$ Gauss-ovom
funkcijom.

Na slici \ref{fig:convolution} su prikazane tri funkcije: četvrtka, zašumljena
četvrtka i neka treća funkcija, zajedno sa odgovarajućim funkcijama $g_n(x)$ za
dvije različite vrijednosti broja $n$ za svaku od te tri funkcije. Za slučaj
četvrtke (slike \ref{fig:rect} - \ref{fig:rect2}), vidi se da funkcionalni niz
$g_n(x)$ jako brzo konvergira ka Gauss-ovoj funkciji. Već za $n=1$ aproksimacija
je dosta dobra, dok za $n=2$ funkcija $g_n(x)$ jako liči na Gauss-ovu. Ponašanje
zašumljene četvrtke (slike \ref{fig:noise} - \ref{fig:noise2}) se može opisati
na sličan način. U slučaju treće funkcije (slike \ref{fig:exp} -
\ref{fig:exp2}), konvergencija je dosta sporija. Na primjer za $n=1$, ne vidi se
nikakva sličnost funkcije $g_n(x)$ sa Gauss-ovom.  Povećanjem broja $n$,
postepeno se uočava sličnost. Međutim, čak i za $n=9$ aproksimacija je lošija
nego u slučaju četvrtke za $n=2$.
%TODO praksa pokazuje za nesimetrične...

\begin{figure}[H]
  \centering
  \begin{tabularx}{\textwidth}{ccc}
    \begin{tabular}{c}
      \fig{0.27}{clt_conv_rect}{Četvrtka}{fig:rect} \\[20pt]
      \fig{0.27}{clt_conv_rect_1}{$n=1$}{fig:rect1} \\
      \fig{0.27}{clt_conv_rect_2}{$n=2$}{fig:rect2}
    \end{tabular}
    &
    \begin{tabular}{c}
      \fig{0.27}{clt_conv_noise}{Zašumljena četvrtka}{fig:noise} \\[20pt]
      \fig{0.27}{clt_conv_noise_1}{$n=1$}{fig:noise1} \\[20pt]
      \fig{0.27}{clt_conv_noise_2}{$n=2$}{fig:noise2}
    \end{tabular}
    &
    \begin{tabular}{c}
      \fig{0.27}{clt_conv_exp}{Treća funkcija}{fig:exp} \\
      \fig{0.27}{clt_conv_exp_1}{$n=1$}{fig:exp1} \\
      \fig{0.27}{clt_conv_exp_2}{$n=9$}{fig:exp2}
    \end{tabular}
  \end{tabularx}
	\caption{Primjeri kontinualnih funkcija i njihovih konvolucija sa samim sobom}
  \label{fig:convolution}
\end{figure}

\subsection{Dokaz centralnog graničnog teorema}

U ovom poglavlju će biti ponuđen dokaz centralnog graničnog teorema. Za potrebe
ovog dokaza potrebno je definirati pojam karakteristične funkcije.

\begin{definition}[Karakteristična funkcija]
  Neka je $X$ slučajna varijabla. Funkcija 
  \begin{equation}
    \varphi_X(u) = E\left[e^{juX}\right]
    = \int_{-\infty}^{\infty} p_X(x) e^{jux} \ \mathrm dx
  \end{equation}
  se naziva karakterističnom funkcijom slučajne varijable $X$.
\end{definition}

Iz definicije se odmah vidi da je karakteristična funkcija
$\varphi_X(u)$\footnote{U literaturi se koriste još i oznake $M_X$ i $\Phi_X$}
jednaka Fourierovoj transformaciji funkcije $p_X(x)$.

\begin{property}
  Ako su $X_i$, $i=1,2,...,n$ nezavisne slučajne varijable i
  $S=X_1+X_2+\cdots+X_n$ njihova suma, onda vrijedi:
  \begin{equation}
    \varphi_S(u) = \prod_{i=1}^{n} \varphi_{X_i}(u)
  \end{equation}
  Specijalno, ako se radi o IID varijablama, onda je
  \begin{equation} \label{eq:char-power}
    \varphi_S(u) = [\varphi_X(u)]^n
  \end{equation}
\end{property}

Dokaz:

\begin{align}
  \varphi_S(u) = E\left[e^{ju(X_1+\cdots+X_n)}\right]
  &= \idotsint\limits_{\mathbb{R}^n}
    p_{X_1,...,X_n}(x_1,...,x_n) e^{ju(X_1+\cdots+X_n)} \ \text d\bm x \nonumber \\
  &= \int_{-\infty}^{\infty} p_{X_1}(x_1)e^{juX_1}\ \mathrm dx_1 \cdots
    \int_{-\infty}^{\infty} p_{X_n}(x_n)e^{juX_n}\ \mathrm dx_n
\end{align}

\begin{property}[Normalna raspodjela]
  Slučajna varijabla iz normalne raspodjele $X \sim \mathcal{N}(\mu, \sigma^2)$
  ima karakterističnu funkciju: %TODO footnote UPUTA?
  \begin{equation} \label{eq:char-norm}
    \varphi_X(u) = e^{j\mu u} e^{-\frac{u^2}{2\sigma^2}}
  \end{equation}
\end{property}

Nakon što su uvedene potrebne definicije i osobine, slijedi dokaz centralnog
graničnog teorema, koji nije potpuno matematski strog, jer su neki detalji
izostavljeni radi jednostavnijeg razumijevanja. \\

\subsection*{Dokaz CLT}

Neka je $\{X_i, i=1,2,...\}$ sekvenca IID slučajnih varijabli sa očekivanjem
$\mu$ i standardnom devijacijom $\sigma$ i neka je
\begin{equation}
  S_n := \frac{1}{\sqrt{n}} \sum_{i=1}^{n} \frac{X_i-\mu}{\sigma}
\end{equation}
Neka se definira pomoćna sekvenca $Y_i := \frac{X_i-\mu}{\sigma}$, za koju
očigledno vrijedi $E[Y_i] = 0$, $\text{Var}[Y_i] = 1$. Ako slučajne varijable
$X_i$ imaju karakteristične funkcije $\varphi_{X_i}(u) = \varphi_X(u)$, onda
slučajne varijable imaju karakteristične funkcije:
\begin{equation} \label{eq:phi_y}
  \varphi_{Y_i}(u) = E\left[e^{ju\frac{X_i-\mu}{\sigma}}\right]
  = \int_{-\infty}^{\infty} p_X(x)e^{ju\frac{x-\mu}{\sigma}} \ \mathrm dx
  = e^{-ju\frac{\mu}{\sigma}} \varphi_X\left(\frac{u}{\sigma}\right)
  = \varphi_Y(u)
\end{equation}
Pogodno je uvesti još jednu pomoćnu sekvencu $Z_{i,n} := Y_i/\sqrt{n}$. Na
sličan način kao u \eqref{eq:phi_y} može se pokazati da je:
\begin{equation}
  \varphi_{Z_{i,n}}(u) = \varphi_{Z,n}(u) := \varphi_Y\left(u\over\sqrt{n}\right)
\end{equation}


Pod određenim uslovima koji ovdje neće biti navedeni, funkcija $\varphi_{Z,n}$
se može aproksimirati Taylorovim razvojem:
\begin{equation} \label{eq:taylor}
  \varphi_{Z,n}(u) = \varphi_Y\left(\frac{u}{\sqrt{n}}\right) 
  = \varphi_Y(0) + \varphi_Y^{'}(0)\frac{u}{\sqrt{n}}
  + \varphi_Y^{''}(0)\frac{u^2}{2n} + o\left(\frac{u^2}{2n}\right),
    \text{ kad } n\to\infty
\end{equation}

Dalje vrijedi:
\begin{align*}
  \varphi_Y(0)
    &= \left.\int_{-\infty}^{\infty} p_{Y_i}(y) e^{juy} \ \text dy\right|_{u=0}
    = \int_{-\infty}^{\infty} p_{Y_i}(y) \ \text dy = 1,\  \forall i \\
  \varphi_Y^{'}(0)
    &= \left.\int_{-\infty}^{\infty} jyp_{Y_i}(y)e^{juy} \ \text dy \right|_{u=0}
    = jE[Y_i] = 0,\ \forall i \\
  \varphi_Y^{''}(0)
    &= \left.-\int_{-\infty}^{\infty} y^2 p_{Y_i}(y)e^{juy} \ \text dy\right|_{u=0}
    = -E[Y_i^2] = -1,\ \forall i
\end{align*}

Formula \eqref{eq:taylor} sada postaje:
\begin{equation}
  \varphi_{Z,n}(u) = 1 - \frac{u^2}{2n} + o\left(\frac{u^2}{2n}\right)
\end{equation}

Pošto vrijedi:
\begin{equation}
  S_n = \sum_{i=1}^{n} Z_{i,n}
\end{equation}
na osnovu formule \eqref{eq:char-power} ova sekvenca će imati karakterističnu
funkciju:
\begin{equation}
  \varphi_{S_n}(u) = \left[\varphi_{Z,n}(u)\right]^n
\end{equation}
U graničnom procesu:
\begin{equation} \label{eq:clt-limit}
  \lim_{n\to\infty} \varphi_{S_n}(u)
  = \lim_{n\to\infty} [\varphi_{Z,n}(u)]^n
  = \lim_{n\to\infty}
    \left(1 - \frac{u^2}{2n} + o\left(\frac{u^2}{2n}\right)\right)^n
    = e^{-\frac{u^2}{2}}
\end{equation}
Na osnovu \eqref{eq:char-norm}, raspodjela koja ima karakterističnu funkciju kao
u \eqref{eq:clt-limit} je upravo normalna $\mathcal{N}(0,1)$. \blacksquare \\

Zanimljivo je razmotriti šta se dešava kada je sekvenca $S_n$ definirana kao
srednja vrijednost uzorka iz \eqref{eq:sample-mean}. Ako se definira pomoćna
sekvenca $W_i = \frac{X_i}{n}$, tada je

\begin{equation}
  S_n = \frac{1}{n}\sum_{i=1}^{\infty} X_i = \sum_{i=1}^{n} W_{i,n}
\end{equation}

Jednostavnim manipulacijama se dolazi do:
\begin{equation}
  W_{i,n} = \frac{X_i}{n} = \frac{\sigma \sqrt{n} Z_{i,n}+\mu}{n}
  = \frac{\sigma}{\sqrt{n}} Z_{i,n} + \frac{\mu}{n}
\end{equation}
gdje su $Y_i$, $Z_{i,n}$ definirane na isti način kao u dokazu CLT.

Primjenom definicije karakteristične funkcije se dobiva:
\begin{equation}
  \varphi_{W_{i,n}}(u) = \varphi_{W,n}(u)
  = e^{j\frac{\mu}{n} u} \varphi_{Z,n}\left(\frac{\sigma u}{\sqrt{n}}\right)
\end{equation}

Dalje vrijedi:

\begin{equation}
  \varphi_{S_n}(u) = [\varphi_{W,n}(u)]^n
  = e^{j\mu u} \left[\varphi_{Z,n}\left(\frac{\sigma u}{\sqrt{n}}\right)\right]^n
\end{equation}

Iskorištavanjem Taylorovog razvoja iz \eqref{eq:taylor}, dolazi se do:
\begin{equation}
  \lim_{n\to\infty}
    \left[\varphi_{Z,n}\left(\frac{\sigma u}{\sqrt{n}}\right)\right]^n
    = \lim_{n\to\infty} \left[1-\frac{\sigma^2u^2}{2n^2} +
      o\left(\frac{\sigma^2u^2}{2n^2}\right)\right]^n
    = 1
\end{equation}
Konačno,
\begin{equation}
  \lim_{n\to\infty} \varphi_{S_n}(u) = e^{ju\mu}
\end{equation}
Primjenom inverzne Fourierove transformacije se dobiva da je
\begin{equation}
  \lim_{n\to\infty} p_{S_n}(s) = \delta(s-\mu)
\end{equation}

Ovaj zaključak se intuitivno podudara sa tvrđenjem zakona velikih brojeva.
Naime, što je uzorak veći, srednja vrijednost uzorka se manje rasipa oko srednje
vrijednosti probabilističkog modela. Za uzorak beskonačne veličine, srednja
vrijednost uzorka se podudara sa matematičkim očekivanjem. \textbf{Drugim
riječima, što je veći broj uzoraka, nesigurnost u tačnost očekivanja
procijenjenog na osnovu tih uzoraka je manja.} %TODO
% ovo prebaciti u LLN?

\subsection{Praktične primjene}

