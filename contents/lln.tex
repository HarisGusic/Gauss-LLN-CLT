Kao u slučaju determinističkih sekvenci, od interesa je posmatrati ponašanje
slučajnih sekvenci (procesa) u beskonačnosti. Međutim, kod slučajnih sekvenci
postoji više različitih pojmova konvergencije.

\begin{definition}[Konvergencija svuda]
  Za slučajnu sekvencu $\{X_n\}$ se kaže da konvergira svuda (sigurno) ka
  slučajnoj varijabli $X$ ako je
  \begin{equation}
    \lim_{n\to\infty} X_n(\omega) = X(\omega)
  \end{equation}
  za svako $\omega \in \Omega$.
\end{definition}

\begin{definition}[Konvergencija skoro svuda]
  Za slučajnu sekvencu $\{X_n\}$ se kaže da konvergira skoro svuda (skoro
  sigurno) ka slučajnoj varijabli $X$ ako je
  \begin{equation}
    \lim_{n\to\infty} X_n(\omega) = X(\omega)
  \end{equation}
  za svako $\omega$ iz nekog skupa $\Lambda$ za koji vrijedi $P(\Lambda)=1$.
\end{definition}

\begin{definition}[Konvergencija po vjerovatnoći]
  Za slučajnu sekvencu $\{X_n\}$ se kaže da konvergira po vjerovatnoći ka
  slučajnoj varijabli $X$ ako za
  svako fiksno $\varepsilon > 0$ vrijedi
  \begin{equation}
    \lim_{n\to\infty} P\left(|X_n - X| \le \varepsilon\right)
  \end{equation}
\end{definition}

TODTOTOTOTOTOTOTOTOTOTODDDDDDDDDDO

\begin{equation} \label{eq:sample-mean}
  S_n = \frac{1}{n} \sum_{i=1}^{n} X_i
\end{equation}


