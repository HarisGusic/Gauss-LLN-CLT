\subsection{Konvergencija slučajnih sekvenci}

Ako su $X_i$, $i=1,2,...$ slučajne varijable, onda se sa $\{X_n\}$ označava
\textit{slučajna sekvenca (proces)}. Razlika između slučajne sekvence i
determinističke sekvence jeste u tome što vrijednost sekvence zavisi od ishoda
nekog eksperimenta.  Ako je $\Omega$ skup svih ishoda nekog eksperimenta, onda
se slučajna sekvenca može zapisati kao $\{X_n(\omega)\}$, pri čemu $\omega \in
\Omega$. Ovaj posljednji zapis će se često koristiti u ostatku poglavlja. Bitno
je primijetiti da sekvenca $X_n(\omega)$ za svaku konkretnu (poznatu) vrijednost
$\omega$ predstavlja determinističku sekvencu, koja se naziva
\textit{realizacija slučajne sekvence}.

Kao u slučaju determinističkih sekvenci, od interesa je posmatrati ponašanje
slučajnih sekvenci (procesa) u beskonačnosti. Međutim, kod slučajnih sekvenci
postoji više različitih pojmova konvergencije. U ovom poglavlju će biti uvedene
neke od njih. 

\begin{definition}[Konvergencija svuda]
  Za slučajnu sekvencu $\{X_n\}$ se kaže da konvergira svuda (sigurno) ka
  slučajnoj varijabli $X$ ako je
  \begin{equation}
    \lim_{n\to\infty} X_n(\omega) = X(\omega)
  \end{equation}
  za svako $\omega \in \Omega$.
\end{definition}

% Primjer TODO rm
%\begin{exmp}
%  Neka je data slučajna sekvenca $X_n(\omega) = \omega^n$, pri čemu je
%  eksperiment opisan uniformnom funkcijom gustine raspodjele $p(\omega)$ na
%  intervalu $[0, 0.5]$, koji se podudara sa skupom ishoda $\Omega$. Ispitati
%  sigurnu konvergenciju sekvence $X_n$.
%\end{exmp}

%Jednostavno se zaključuje da je
%\begin{equation}
%  \lim_{n\to\infty} X_n(\omega)
%  = \lim_{n\to\infty} \omega^n = 0,\ \forall \omega \in \Omega
%\end{equation}

%Dakle, sekvenca $X_n(\omega)$ konvergira svuda ka $X=0$. U ovom
%primjeru sekvenca konvergira ka istom broju za svaku vrijednost $\omega$, tj.
%nezavisno od ishoda eksperimenta.

% Primjer
\begin{exmp}
  Neka je slučajna sekvenca $X_n(\omega) = \frac{\omega n}{n+1}$, $\omega \in
  \Omega = \mathbb{R}$. Ispitati sigurnu konvergenciju.
\end{exmp}

Vrijedi
\begin{equation}
  \lim_{n\to\infty} X_n(\omega) = \omega \lim_{n\to\infty}\frac{n}{n+1} = \omega
\end{equation}

Dakle, sekvenca $X_n(\omega)$ konvergira svuda ka slučajnoj varijabli $X(\omega)
= \omega$. U ovom primjeru slučajna sekvenca $X_n$ konvergira za svako $\omega$.
Međutim granična vrijednost zavisi od $\omega$, tj. od ishoda eksperimenta, što
je opisano slučajnom varijablom $X$.

\begin{exmp} \label{ex:converge-surely}
  Neka je $X_n(\omega) = \omega^n$, pri čemu je $\Omega= \mathbb{R}$ eksperiment
  opisan uniformnom PDF:
  $$p(\omega) = 1_{[0,1]}(\omega) = \begin{cases}
      1, \omega \in [0,1]\ \\ 0,\ \text{\normalfont inače}
    \end{cases}
  $$
  Ispitati sigurnu konvergenciju sekvence $X_n$.
\end{exmp}

Jasno je da sekvenca $\omega^n$ konvergira ka $X=0$ za $\omega \in [0,1)$, te ka
$X=1$ za $\omega = 1$. Međutim, za sve ostale vrijednosti $\omega$, sekvenca
divergira. Dakle, sekvenca $X_n$ ne konvergira svuda.\\

Međutim, konvergencija svuda u praksi često nije potrebna. Na primjer, nije od
velikog interesa da li će sekvenca $X_n$ iz prethodnog primjera konvergirati za
$\omega \in \mathbb{R}\setminus[0,1]$, jer je taj događaj nemoguć. Zato se uvodi
pojam konvergencije skoro svuda.

\begin{definition}[Konvergencija skoro svuda]
  Za slučajnu sekvencu $\{X_n\}$ se kaže da konvergira skoro svuda (skoro
  sigurno, ili sa vjerovatnoćom 1) ka slučajnoj varijabli $X$ ako je
  \begin{equation}
    \lim_{n\to\infty} X_n(\omega) = X(\omega)
  \end{equation}
  za svako $\omega$ iz nekog skupa $\Lambda$ za koji vrijedi
  $\pr(\omega\in\Lambda)=1$.
\end{definition}

Sekvenca iz primjera \ref{ex:converge-surely} konvergira skoro sigurno jer
sekvenca konvergira za $\forall \omega \in \Lambda = [0,1]$, pri čemu je
$\pr(\omega \in [0,1]) = 1$.

\begin{definition}[Konvergencija po vjerovatnoći]
  Za slučajnu sekvencu $\{X_n\}$ se kaže da konvergira po vjerovatnoći ka
  slučajnoj varijabli $X$ ako za svako fiksno $\varepsilon > 0$ vrijedi
  \footnote{
    Ovaj uslov se često (ekvivalentno) navodi kao
    $\lim_{n\to\infty} \pr(|X_n-X|>\varepsilon) = 0$
  }
  \begin{equation}
    \lim_{n\to\infty} \pr\left(|X_n - X| \le \varepsilon\right) = 1
  \end{equation}
\end{definition}

Grubo rečeno, sekvenca koja konvergira po vjerovatnoći je takva da relativno
veliki broj realizacija sekvence dovoljno dobro aproksimira slučajnu varijablu
$X$ u beskonačnosti. Ekvivalentno, to znači da relativno mali broj realizacija
nedovoljno dobro aproksimira slučajnu varijablu $X$ u beskonačnosti. Pojam
\textit{dovoljno dobro} ovdje znači u odnosu na odabrano $\varepsilon$. Iz skoro
sigurne konvergencije slijedi konvergencija po vjerovatnoći, dok obrat ne važi.
Intuitivno, konvergencija po vjerovatnoći znači "Svaka realizacija vjerovatno
konvergira", dok konvergencija skoro svuda znači "Praktično je sigurno da svaka
realizacija konvergira". \\

Prethodne definicije su potrebne za formulaciju zakona velikih brojeva, dok će
sljedeća definicija biti od koristi za centralni granični teorem.  U nastavku će
sa $P_X(x)$ biti označena funkcija raspodjele vjerovatnoće slučajne varijable
$X$.

\begin{definition}[Konvergencija po distribuciji]
  Za slučajnu sekvencu $\{X_n\}$ se kaže da konvergira po raspodjeli ka
  slučajnoj varijabli $X$ ako je
  \begin{equation}
    \lim_{n\to\infty} P_{X_n}(x) = P_X(x)
  \end{equation}
  u svakoj tački $x$ u kojoj je funkcija $P_X$ neprekidna.
\end{definition}

Korisno je istaknuti da su svi prethodni konvergencije navedeni od najstrožijeg
ka najblažem.

Prije nego što se formuliše zakon velikih brojeva, potrebno je
definirati srednju vrijednost uzorka (engl. \textit{sample mean}).

\begin{definition}[Srednja vrijednost uzorka]

  Neka je $\{X_n\}$ slučajna sekvenca. Slučajna varijabla $S_n$
  definirana kao 
  \begin{equation} \label{eq:sample-mean}
    S_n = \frac{1}{n} \sum_{i=1}^{n} X_i
  \end{equation}
  se naziva srednja vrijednost uzorka $X_1,X_2,...,X_n$.

\end{definition}
%
Sljedeća pitanja su od praktičnog interesa:

\begin{itemize}
  \item Ponašanje empirijski dobivene srednje vrijednosti $S_n$, konvergencija i
    na koji način konvergencija zavisi od sekvence $\{X_n\}$.
  \item Statističke osobine malih varijacija sekvence $S_n$ oko srednje
    vrijednosti.\footnote{Često su od interesa su i statističke osobine velikih
    varijacija oko srednje vrijednosti, što neće biti razmatrano u ovom radu.}
\end{itemize}

Odgovor na prvo pitanje daju zakoni velikih brojeva, dok odgovor na drugo
pitanje daje centralni granični teorem koji će biti obrađen u sljedećem
poglavlju. U literaturi se najčešće susreću dva zakona velikih brojeva: slabi i
jaki. Dodatno, oba zakona se susreću u više verzija. U nastavku će ovi zakoni
biti formulisani način koji najbolje ilustruje razliku između njih.

\begin{theorem}[Slabi zakon velikih brojeva]
  Neka je $\{X_n\}$ sekvenca slučajnih varijabli sa očekivanjem $\mu$ i konačnom
  i ograničenom varijansom $\sigma_n^2 \le \sigma^2 < \infty$ i neka je
  $\{S_n\}$ sekvenca srednjih vrijednosti uzoraka . Tada za svako
  $\varepsilon>0$
  \begin{equation}
    \lim_{n\to\infty} \pr(|S_n - \mu| \le \varepsilon) = 1
  \end{equation}
  Drugim riječima, sekvenca $\{S_n\}$ konvergira ka $\mu$ po vjerovatnoći.
\end{theorem}
Druga verzija ovog teorema koja se češće susreće u literaturi je manje općenita,
a razlikuje se od prethodne u tome što se za $\{S_n\}$ pretpostavlja da je IID.
Jedan heuristički dokaz ovog teorema će biti dat u poglavlju \ref{sec:clt-lln}.

\begin{theorem}[Jaki zakon velikih brojeva]
  Neka je $\{X_n\}$ sekvenca IID slučajnih varijabli sa očekivanjem $\mu$ i
  konačnom varijansom. Tada sekvenca $\{S_n\}$ srednjih vrijednosti uzoraka
  konvergira skoro sigurno ka $\mu$.
\end{theorem}

Postoji i uslov koji garantuje važenje ovog teorema i za sekvence koje nisu
identično raspodijeljene, ali se ovdje neće razmatrati.

Sada će prethodni koncepti biti ilustrirani kroz primjer.

\subsection{Generalizacije i srodne tvdnje}

