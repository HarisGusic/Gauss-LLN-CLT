U ovom poglavlju će biti objašnjen jedan od najznačajnijih rezultata u oblasti
vjerovatnoće. Centralni granični teorem (engl.  \textit{central limit theorem -
CLT}) opisuje ponašanje sume velikog broja slučajnih varijabli. Prije nego što
se formuliše teorem, biće napravljeno uvodno razmatranje na tu temu.

\subsection{Raspodjela sume slučajnih varijabli} \label{sec:clt-suma}

Neka su $K$ i $L$ dvije diskretne slučajne varijable sa masenim funkcijama
raspodjele (PMF) $p_K(k)$ i $p_L(l)$ respektivno. Neka je potrebno naći PMF
$p_M(m)$ slučajne varijable $M=K+L$.
%
\begin{align*}
  p_M(m) = \pr(M=m) = \pr(k\in\mathbb{Z}, l=m-k)
  = \sum_{k=-\infty}^{\infty} p_{K,L}(k,m-k)
\end{align*}

Ako su varijable $K$ i $L$ međusobno nezavisne, onda vrijedi $p_{K,L}(k,l) =
p_K(k)p_L(l)$, te se prethodni izraz može dodatno pojednostaviti:
%
\begin{align}
  p_M(m) = \sum_{k=-\infty}^{\infty} p_K(k)p_L(m-k) = p_K * p_L (m)
\end{align}

Dakle, suma nezavisnih diskretnih slučajnih varijabli će imati PMF koja je
jednaka diskretnoj konvoluciji PMF tih varijabli. Sličan zaključak vrijedi i za
slučaj kontinualnih varijabli.

\begin{theorem}

  Ako su $X_1,X_2,...,X_n$ međusobno nezavisne slučajne varijable sa funkcijama
  gustine vjerovatnoće $p_{X_i}$, $i=1,...,n$ respektivno, onda njihova suma $S$
  ima funkciju gustine vjerovatnoće:
  $$p_S(s) = p_{X_1} * p_{X_2} * \cdots * p_{X_n} (s)$$

\end{theorem}

Prethodno razmatranje će sada biti intuitivno objašnjeno na primjeru.  Radi
jednostavnosti, razmatranje će biti ograničeno na diskretni slučaj.

\begin{exmp} \label{ex:novcic}
  
Neka su $X_i$ slučajne varijable koje uzimaju vrijednosti iz $\{0,1\}$ zavisno
od ishoda bacanja nepravednog novčića, pri čemu je $p_X(1)= p$, $p_X(0)=q=1-p$.
Potrebno je odrediti broj bacanja $K$ u kojima je $X_i=1$, ako je novčić bačen
$n$ puta.

\end{exmp}

Traženi broj $K$ također predstavlja slučajnu varijablu i može se izraziti
kao:
$$K = \sum_{i=1}^{n} X_i$$
\\

Pošto su slučajne varijable $X_i$ IID, PMF od $K$ se dobija konvolucijom:
$$p_K(k) = \underbrace{p_X * p_X * \cdots * p_X}_{n\text{ puta}}(k)$$

Jednostavan način da se odredi ova konvolucija jeste korištenjem
$\mathcal{Z}$ transformacije.\footnote{\ $p_K(k)$ se može odrediti i čisto
kombinatornim pristupom.}
Naime, vrijedi:
\begin{equation}
  \mathcal{Z}\{p_K(k)\}
  = \left[\mathcal{Z}\{p_X(k)\}\right]^n
  = (q+pz^{-1})^n
  = \sum_{k=0}^{n} \binom{n}{k} p^k z^{-k} q^{n-k}
\end{equation}

Direktnom primjenom inverzne $ \mathcal{Z}$ transformacije se dobiva:
\begin{equation}
  p_K(k) = \binom{n}{k} p^kq^{n-k}
\end{equation}

Radi jednostavnosti, u nastavku će biti pretpostavljeno da je novčić pravedan,
tj. $p=q=1/2$, pa će odgovarajuća PMF biti:
\begin{equation} \label{eq:binom-PMF}
  p_K(k) = \frac{1}{2^n} \binom{n}{k}
\end{equation}

Na slikama \ref{fig:binom} je prikazan izgled PMF iz \eqref{eq:binom-PMF} za
razne vrijednosti $n$.  Na slici \ref{fig:binom:a} je prikazana PMF $p_K(k)$, za
jedno bacanje, koja se naravno podudara sa $p_X(k)$. Ovdje postoje dva moguća
ishoda, sa jednakom vjerovatnoćom koja iznosi 0.5. Ako se novčić baci drugi put
(slika \ref{fig:binom:b}), mogući ishodi za $K$ su $0, 1, 2$, pri čemu je ishod
$1$ dvostruko vjerovatniji od ostalih, jer se može dobiti na dva načina
($1=0+1=1+0$), dok se ostali mogu dobiti na samo jedan način ($0=0+0$, $2=1+1$).
Ako se novčić baci treći put (slika \ref{fig:binom:c}), mogući ishodi su
$0,1,2,3$, pri čemu vrijednosti 1 i 2 imaju najveću vjerovatnoću. Na slikama
\ref{fig:binom:d} i \ref{fig:binom:e} su prikazane PMF za $n=8$ i $n=50$
respektivno. Već na slici \ref{fig:binom:d} se može primijetiti da je PMF
skoncentrisana oko sredine intervala, dok se na slici \ref{fig:binom:e} vidi da
PMF aproksimira Gaussovu funkciju. Ova pojava se može formalno iskazati
\textit{De Moivre-Laplace}-ovim teoremom koji tvrdi da je:

\begin{equation}
  \binom{n}{k}p^kq^{n-k} \simeq \frac{1}{\sqrt{2\pi npq}}
  e^{-\frac{(k-np)^2}{2npq}}
\end{equation}
Dakle, binomna raspodjela aproksimira Gaussovu funkciju za veliko $n$. \\

Prethodni primjer je dosta regularan zbog činjenice da je novčić pravedan, pa su
sve PMF simetrične oko neke centralne tačke.

\begin{figure}[H]
  \centering
    \fig{0.31}{clt_binom_1}{$n=1$}{fig:binom:a}
    \fig{0.31}{clt_binom_2}{$n=2$}{fig:binom:b}
    \fig{0.31}{clt_binom_3}{$n=3$}{fig:binom:c}
    \fig{0.31}{clt_binom_8}{$n=8$}{fig:binom:d}
    \fig{0.31}{clt_binom_50}{$n=50$}{fig:binom:e}
	\caption{Binomna raspodjela i De Moivre-Laplace aproksimacija}
  \label{fig:binom}
\end{figure}

\subsection{Formulacija centralnog graničnog teorema} \label{sec:clt-formulation}

Postoje različite formulacije centralnog graničnog teorema, ali ovdje će biti
data jedna od najjednostavnijih, mada ne i najopćenitija.

\begin{theorem} \label{th:clt}
  Neka je $\{X_i, i=1,2,...\}$ sekvenca IID slučajnih varijabli sa
  konačnim očekivanjem $\mu$ i konačnom varijansom $\sigma^2$. Tada
  sekvenca
  \begin{equation}
    S_n = \frac{1}{\sqrt{n}} \sum_{i=1}^{n} \frac{X_i-\mu}{\sigma}
  \end{equation}
  konvergira po distribuciji ka normalnoj raspodjeli $\mathcal{N}(0, 1)$.
\end{theorem}

Nažalost, način kako je definirana sekvenca $S_n$ nije najbolji oblik za
intuitivno shvatanje teorema. Prirodnije bi bilo sekvencu $S_n$ definirati kao
$S_n=\sum_{i=1}^{n} X_i$, bez suvišnih oznaka. Međutim, ukoliko se koristi takva
definicija, nastaje problem u tome što takva sekvenca neće konvergirati po
distribuciji. Raspodjela takve sekvence će sa povećanjem $n$ bolje i bolje
aproksimirati \textbf{neku} Gaussovu raspodjelu $\mathcal{N}(n\mu,n\sigma^2)$
koja zavisi od $n$, ali očigledno ovakva sekvenca raspodjela neće težiti ka
istoj raspodjeli. Još jedan prirodan način definiranja sekvence $S_n$ bi bio kao
kao srednja vrijednost uzorka sekvence $X_n$. Međutim ovdje bi problem napravio
zakon velikih brojeva uslijed kojeg će sa povećanjem $n$ varijansa težiti u
nulu. Ovaj slučaj će biti dodatno razmotren u poglavlju \ref{sec:dokaz-clt}.
Radi dodatne ilustracije, u slučaju definiranja $S_n = \left(\sum_{i=1}^n
X_i\right)/\sqrt{n}$, varijansa bi konvergirala, ali bi očekivanje $\mu
\sqrt{n}$ divergiralo. Međutim, uprkos ovih problema, često se susreću
formulacije koje sekvencu $S_n$ definiraju na prethodno navedene načine, no
takve formulacije su manje precizne. \\

Centralni granični teorem ima nekoliko interesantnih posljedica koje su
objašnjene u nastavku.

\setcounter{corollary}{0}
\begin{corollary}

  Suma velikog broja IID slučajnih varijabli sa očekivanjem $\mu$ i varijansom
  $\sigma^2$ imati raspodjelu koja aproksimira $\mathcal{N}(n\mu, n\sigma^2)$,
  nezavisno od raspodjele varijabli koje učestvuju u sumi. 

\end{corollary}

Ovakva interpretacija teorema je jednostavnija i intuitivnija nego sama
formulacija teorema koja je data iznad.

\begin{corollary}
  Konvolucijom pozitivne funkcije sa samom sobom mnogo puta dobiva se približno
  Gaussova funkcija.
\end{corollary}

Ovo je već ilustrovano u primjeru \ref{ex:novcic}, mada je tamo bila razmatrana
raspodjela u diskretnom slučaju koja je dosta regularna i simetrična. U nastavku
će isti princip biti prikazan na nekoliko različitih kontinualnih funkcija. Neka
je $f(x)$ neka nenegativna funkcija dovoljno "lijepih" osobina, i neka je:

\begin{equation}
  g_n(x) = f^{*n}(x) := \underbrace{f*f\cdots*f}_{n\text{ puta}}\ (x)
\end{equation}
%
Također, neka je $\widetilde{g}_n(x)$ aproksimacija funkcije $g_n(x)$ Gaussovom
funkcijom. \\

Na slici \ref{fig:convolution} su prikazane tri funkcije: četvrtka, zašumljena
četvrtka i neka treća funkcija, zajedno sa odgovarajućim funkcijama $g_n(x)$ za
dvije različite vrijednosti broja $n$ za svaku od te tri funkcije. Za slučaj
četvrtke (slike \ref{fig:rect} - \ref{fig:rect2}), vidi se da funkcionalni niz
$g_n(x)$ jako brzo konvergira ka Gaussovoj funkciji. Već za $n=1$ aproksimacija
je dosta dobra, dok za $n=2$ funkcija $g_n(x)$ jako liči na Gaussovu. Ponašanje
zašumljene četvrtke (slike \ref{fig:noise} - \ref{fig:noise2}) se može opisati
na sličan način. U slučaju treće funkcije (slike \ref{fig:exp} -
\ref{fig:exp2}), konvergencija je dosta sporija. Na primjer za $n=1$, ne vidi se
nikakva sličnost funkcije $g_n(x)$ sa Gaussovom.  Povećanjem broja $n$,
postepeno se uočava sličnost. Međutim, čak i za $n=9$ aproksimacija je lošija
nego u slučaju četvrtke za $n=2$.
%
\begin{figure}[H]
  \centering
  \begin{tabularx}{\textwidth}{ccc}
    \begin{tabular}{c}
      \fig{0.265}{clt_conv_rect}{Četvrtka}{fig:rect} \\[20pt]
      \fig{0.265}{clt_conv_rect_1}{$n=1$}{fig:rect1} \\
      \fig{0.265}{clt_conv_rect_2}{$n=2$}{fig:rect2}
    \end{tabular}
    &
    \begin{tabular}{c}
      \fig{0.265}{clt_conv_noise}{Zašumljena četvrtka}{fig:noise} \\[20pt]
      \fig{0.265}{clt_conv_noise_1}{$n=1$}{fig:noise1} \\[20pt]
      \fig{0.265}{clt_conv_noise_2}{$n=2$}{fig:noise2}
    \end{tabular}
    &
    \begin{tabular}{c}
      \fig{0.265}{clt_conv_exp}{Treća funkcija}{fig:exp} \\
      \fig{0.265}{clt_conv_exp_1}{$n=1$}{fig:exp1} \\
      \fig{0.265}{clt_conv_exp_2}{$n=9$}{fig:exp2}
    \end{tabular}
  \end{tabularx}
	\caption{Primjeri kontinualnih funkcija i njihovih konvolucija sa samim sobom}
  \label{fig:convolution}
\end{figure}

\subsection{Dokaz centralnog graničnog teorema} \label{sec:dokaz-clt}

U ovom poglavlju će biti ponuđen dokaz centralnog graničnog teorema. Za potrebe
ovog dokaza potrebno je definirati pojam karakteristične funkcije.

\subsubsection{Pojam karakteristične funkcije}

\begin{definition}[Karakteristična funkcija]
  Neka je $X$ slučajna varijabla. Funkcija 
  \begin{equation}
    \varphi_X(u) := E\left[e^{juX}\right]
    = \int_{-\infty}^{\infty} p_X(x) e^{jux} \ \mathrm dx
  \end{equation}
  se naziva karakterističnom funkcijom slučajne varijable $X$.
\end{definition}

Iz definicije se odmah vidi da je karakteristična funkcija
$\varphi_X(u)$\footnote{U literaturi se koriste još i oznake $M_X$ i $\Phi_X$}
jednaka Fourierovoj transformaciji funkcije $p_X(x)$.\footnote{ Ovakva
  definicija se razlikuje od uobičajene definicije Fourierove transformacije u
inženjerskim primjenama, gdje se varijabla transformacije uzima sa različitim
predznakom.}

\begin{property}
  Ako su $X_i$, $i=1,2,...,n$ nezavisne slučajne varijable i
  $S=X_1+X_2+\cdots+X_n$ njihova suma, onda vrijedi:
  \begin{equation}
    \varphi_S(u) = \prod_{i=1}^{n} \varphi_{X_i}(u)
  \end{equation}
  Specijalno, ako se radi o IID varijablama, onda je
  \begin{equation} \label{eq:char-power}
    \varphi_S(u) = [\varphi_X(u)]^n
  \end{equation}
\end{property}

Dokaz:

\begin{align}
  \varphi_S(u)
  &= E\left[e^{ju(X_1+\cdots+X_n)}\right] \\
  &= \idotsint\limits_{\mathbb{R}^n}
    p_{X_1,...,X_n}(x_1,...,x_n) e^{ju(X_1+\cdots+X_n)} \ \text d\bm x \nonumber \\
  &= \int_{-\infty}^{\infty} p_{X_1}(x_1)e^{juX_1}\ \mathrm dx_1 \cdots
    \int_{-\infty}^{\infty} p_{X_n}(x_n)e^{juX_n}\ \mathrm dx_n
\end{align}

\begin{property}[Normalna raspodjela]
  Slučajna varijabla iz normalne raspodjele $\mathcal{N}(\mu, \sigma^2)$ ima
  karakterističnu funkciju:
  \begin{equation} \label{eq:char-norm}
    \varphi_X(u) = e^{j\mu u} e^{-\frac{u^2}{2\sigma^2}}
  \end{equation}
\end{property}

Odavde se lako provjerava da će suma međusobno nezavisnih Gaussovih varijabli
biti Gaussova, što je već pokazano u poglavlju \ref{sec:gauss-multi}. \\

Nakon što su uvedene potrebne definicije i osobine, slijedi dokaz centralnog
graničnog teorema, koji nije potpuno matematski strog, jer su neki detalji
izostavljeni radi jednostavnijeg razumijevanja. \\

\subsubsection{Dokaz}

Neka je $\{X_i, i=1,2,...\}$ sekvenca IID slučajnih varijabli sa očekivanjem
$\mu$ i standardnom devijacijom $\sigma$ i neka je
\begin{equation}
  S_n := \frac{1}{\sqrt{n}} \sum_{i=1}^{n} \frac{X_i-\mu}{\sigma}
\end{equation}
Neka se definira pomoćna sekvenca $Y_i := \frac{X_i-\mu}{\sigma}$, za koju
očigledno vrijedi $E[Y_i] = 0$, $\text{Var}[Y_i] = 1$. Ako slučajne varijable
$X_i$ imaju karakteristične funkcije $\varphi_{X_i}(u) = \varphi_X(u)$, onda
slučajne varijable imaju karakteristične funkcije:
\begin{equation} \label{eq:phi_y}
  \varphi_{Y_i}(u) = E\left[e^{ju\frac{X_i-\mu}{\sigma}}\right]
  = \int_{-\infty}^{\infty} p_X(x)e^{ju\frac{x-\mu}{\sigma}} \ \mathrm dx
  = e^{-ju\frac{\mu}{\sigma}} \varphi_X\left(\frac{u}{\sigma}\right)
  = \varphi_Y(u)
\end{equation}
Pogodno je uvesti još jednu pomoćnu sekvencu $Z_{i,n} := Y_i/\sqrt{n}$. Na
sličan način kao u \eqref{eq:phi_y} može se pokazati da je:
\begin{equation}
  \varphi_{Z_{i,n}}(u)
  = \varphi_{Z,n}(u) := \varphi_Y\left(\frac{u}{\sqrt{n}}\right)
\end{equation}

Pod određenim uslovima koji ovdje neće biti navedeni, funkcija $\varphi_{Z,n}$
se može aproksimirati Taylorovim razvojem:
\begin{equation} \label{eq:taylor}
  \varphi_{Z,n}(u) = \varphi_Y\left(\frac{u}{\sqrt{n}}\right) 
  = \varphi_Y(0) + \varphi_Y^{'}(0)\frac{u}{\sqrt{n}}
  + \varphi_Y^{''}(0)\frac{u^2}{2n} + o\left(\frac{u^2}{2n}\right),
    \text{ kad } n\to\infty
\end{equation}
%
Dalje vrijedi:
\begin{align*}
  \varphi_Y(0)
    &= \left.\int_{-\infty}^{\infty} p_{Y_i}(y) e^{juy} \ \text dy\right|_{u=0}
    = \int_{-\infty}^{\infty} p_{Y_i}(y) \ \text dy = 1,\  \forall i \\
  \varphi_Y^{'}(0)
    &= \left.\int_{-\infty}^{\infty} jyp_{Y_i}(y)e^{juy} \ \text dy \right|_{u=0}
    = jE[Y_i] = 0,\ \forall i \\
  \varphi_Y^{''}(0)
    &= \left.-\int_{-\infty}^{\infty} y^2 p_{Y_i}(y)e^{juy} \ \text dy\right|_{u=0}
    = -E[Y_i^2] = -1,\ \forall i
\end{align*}
%
Formula \eqref{eq:taylor} sada postaje:
\begin{equation}
  \varphi_{Z,n}(u) = 1 - \frac{u^2}{2n} + o\left(\frac{u^2}{2n}\right)
\end{equation}
%
Pošto vrijedi:
\begin{equation}
  S_n = \sum_{i=1}^{n} Z_{i,n}
\end{equation}
na osnovu formule \eqref{eq:char-power} ova sekvenca će imati karakterističnu
funkciju:
\begin{equation}
  \varphi_{S_n}(u) = \left[\varphi_{Z,n}(u)\right]^n
\end{equation}
U graničnom procesu:
\begin{align} \label{eq:clt-limit}
  \lim_{n\to\infty} \varphi_{S_n}(u)
  &= \lim_{n\to\infty} [\varphi_{Z,n}(u)]^n \nonumber \\
  &= \lim_{n\to\infty}
    \left(1 - \frac{u^2}{2n} + o\left(\frac{u^2}{2n}\right)\right)^n
    = e^{-\frac{u^2}{2}}
\end{align}
Na osnovu \eqref{eq:char-norm}, raspodjela koja ima karakterističnu funkciju kao
u \eqref{eq:clt-limit} je upravo normalna $\mathcal{N}(0,1)$. $\blacksquare$ \\

\subsection{Veza sa zakonom velikih brojeva} \label{sec:clt-lln}

Zanimljivo je razmotriti šta se dešava kada je sekvenca $S_n$ definirana kao
sekvenca srednjih vrijednosti uzoraka iz \eqref{eq:sample-mean}. Naime, ako se
definira pomoćna sekvenca $W_i = \frac{X_i}{n}$, tada je

\begin{equation}
  S_n = \frac{1}{n}\sum_{i=1}^{n} X_i = \sum_{i=1}^{n} W_{i,n}
\end{equation}
%
Jednostavnim manipulacijama se dolazi do:
\begin{equation}
  W_{i,n} = \frac{X_i}{n} = \frac{\sigma \sqrt{n} Z_{i,n}+\mu}{n}
  = \frac{\sigma}{\sqrt{n}} Z_{i,n} + \frac{\mu}{n}
\end{equation}
gdje su $Y_i$, $Z_{i,n}$ definirane na isti način kao u dokazu CLT.
%
Primjenom definicije karakteristične funkcije se dobiva:
\begin{equation}
  \varphi_{W_{i,n}}(u) = \varphi_{W,n}(u)
  = e^{j\frac{\mu}{n} u} \varphi_{Z,n}\left(\frac{\sigma u}{\sqrt{n}}\right)
\end{equation}
%
Dalje vrijedi:

\begin{equation}
  \varphi_{S_n}(u) = [\varphi_{W,n}(u)]^n
  = e^{j\mu u} \left[\varphi_{Z,n}\left(\frac{\sigma u}{\sqrt{n}}\right)\right]^n
\end{equation}
%
Iskorištavanjem Taylorovog razvoja iz \eqref{eq:taylor}, dolazi se do:
\begin{equation}
  \lim_{n\to\infty}
    \left[\varphi_{Z,n}\left(\frac{\sigma u}{\sqrt{n}}\right)\right]^n
    = \lim_{n\to\infty} \left[1-\frac{\sigma^2u^2}{2n^2} +
      o\left(\frac{\sigma^2u^2}{2n^2}\right)\right]^n
    = 1
\end{equation}
Konačno,
\begin{equation}
  \lim_{n\to\infty} \varphi_{S_n}(u) = e^{ju\mu}
\end{equation}
Inverzna Fourierova transformacija ove funkcije je $\delta(s-\mu)$. Intuitivno
je da će se raspodjela sekvence srednjih vrijednosti uzoraka degenerisati u ovu
funkciju. Prikaz ovoga je dat na slici \ref{fig:clt-degen}. Pri tome je sa
$\widetilde{p_{S_n}}(x)$ označena aproksimacija raspodjele sekvence $S_n$
normalnom raspodjelom. Sekvenca $X_n$ nad kojom je definirana sekvenca $S_n$ ima
standardnu normalnu raspodjelu.

\begin{figure}[H]
  \centering
  \includegraphics[width=0.4\textwidth]{clt_degen}
	\caption{Degeneracija normalne raspodjele sa smanjenjem varijanse}
  \label{fig:clt-degen}
\end{figure}

Prethodno izvođenje se ne smije shvatiti kao matematski strogo, međutim dobiveni
zaključak ima veliku ilustrativnu moć.\footnote{Matematski strog tretman se može
pronaći u knjizi \cite{fristedt2013modern} na stranici 273} Ovaj zaključak se
intuitivno podudara sa tvrđenjem zakona velikih brojeva.  Naime, što je uzorak
veći, srednja vrijednost uzorka se manje rasipa oko srednje vrijednosti
probabilističkog modela. Za uzorak beskonačne veličine, srednja vrijednost
uzorka se podudara sa matematičkim očekivanjem.  \textit{Drugim riječima, što je
veći uzorak, nesigurnost u tačnost procijenjenog očekivanja na osnovu tog uzorka
je manja.}

\subsection{Praktične primjene}

Centralni granični teorem objašnjava zašto se normalna raspodjela susreće toliko
često u prirodi. Naime, česti su fenomeni koji nastaju superpozicijom više
međusobno nezavisnih fenomena. Na primjer, termalno (haotično) kretanje
elektrona se može smatrati slučajnim. Amplituda termalnog šuma koji se javlja u
električnim krugovima se dobiva superpozicijom doprinosa svakog elektrona. Kao
posljedica CLT, distribucija amplitude ovakvog šuma će biti približno normalna.
Može se navesti još nekoliko primjera. \footnote{Primjeri su preuzeti iz knjige
\cite{smith2003digital}}.


Još jedna interesantna posljedica CLT je da će svjetlosni signal u obliku
četvrtke koji ulazi u optički vodič, na izlazu iz optičkog vodiča biti
"razliven". Izlazni signal će biti približno Gaussov, što se objašnjava
činjenicom da svjetlosni zraci prelaze različite puteve, zbog toga što zraci
ulazi u vodič pod različitim uglovima. Oblik izlaznog signala zavisi od dužine
vodiča. Ova pojava ima određene implikacije na brzinu prenosa informacija kroz
optički vod.

CLT ima primjenu i u digitalnom procesiranju signala. Naime, kaskada
\textit{moving average} filtera će imati impulsni odziv oblika Gaussove
funkcije. Ovo je posljedica osobine višestruke konvolucije koja je ilustrovana u
poglavlju \ref{sec:clt-formulation}. Ovakav filter za mnoge primjene ima bolje
osobine od običnog moving average filtera.
Konačno, CLT se primjenjuje i za kompjutersko generisanje slučajnih sekvenci
koje imaju normalnu raspodjelu.

\subsection{Generalizacije i srodne tvrdnje}

Kao što je već rečeno, oblik CLT formulisan u ovom radu nije najopćenitiji.
Postoje generalizacije koje ne zahtijevaju da slučajne varijable koje se
sumiraju imaju jednaku varijansu. Naime, postoje formulacije teorema koje
garantuju da zaključak teorema vrijedi ako su slučajne varijable koje se
sumiraju "slabo" međusobno korelirane. Ovdje neće biti objašnjeno šta takva
"slaba" korelacija podrazumijeva.
%
Za praktične primjene, može se smatrati da CLT vrijedi kada:

\begin{enumerate}
  \item Slučajne varijable imaju konačno očekivanje i konačnu varijansu,
  \item Slučajne varijable su "slabo" korelirane,
  \item Doprinos svake od varijabli u sumi nije previše velik.
\end{enumerate}

Postoje i teoremi koji daju procjenu brzine konvergencije ka normalnoj
raspodjeli, no u većini praktičnih primjena se konvergencija je \textit{veoma
brza}. \\

U ovom radu, CLT je uveden postepeno na primjeru sumiranja diskretnih varijabli.
Naveden je \textit{De Moivre-Laplace} teorem koji govori da će binomna
raspodjela aproksimirati normalnu raspodjelu. Međutim u primjeru koji je
razmatran, slučaj je da i PMF odgovarajuće diskretne slučajne sekvence
aproksimira Gaussovu funkciju. Ovo nije uvijek slučaj. Naime, distribucija
sekvence (preciznije CDF) srednjih vrijednosti uzoraka $\{S_n\}$ diskretne
sekvence $\{X_n\}$ će također konvergirati ka normalnoj distribuciji, ali to ne
znači da će i PMF od $S_n$ konvergirati. Pokazuje se da u slučaju kada su sve
vrijednosti sekvence $X_n$ uniformno distribuirane, i PMF sekvence $S_n$ će
konvergirati.

CLT se može generalizirati i za sekvence slučajnih vektora, pri čemu će suma
velikog broja slučajnih vektora (uz odgovarajuće uvjete) biti približno
normalna.

