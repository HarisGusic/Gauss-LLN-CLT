\newcommand*{\vecrow}[2]{\left[\begin{array}{cc}#1&#2\end{array}\right]}
\newcommand*{\veccol}[2]{\left[\begin{array}{c}#1\\#2\end{array}\right]}

U ovom poglavlju će biti obrađena normalna raspodjela, koja je važna zato što se
u praksi jako često susreće.

\subsection{Funkcija gustine vjerovatnoće}

\begin{definition}
  Za slučajnu varijablu $X$ se kaže da je normalno raspodijeljena ako ima
  funkciju gustine vjerovatnoće (PDF):
  \begin{equation} \label{eq:normal}
    p(x) = \frac{1}{\sqrt{2\pi\sigma^2}} e^{-\frac{(x-\mu)^2}{2\sigma^2}}
  \end{equation}
  Piše se $X \sim \mathcal{N}(\mu,\sigma^2)$.
\end{definition}

Ovakva slučajna varijabla se naziva još i \textit{Gaussova}. Lako se pokazuje da
ovako definirana Gaussova slučajna varijabla ima očekivanu vrijednost $\mu$ i
varijansu $\sigma^2$ (tj. standardnu devijaciju $\sigma$).\footnote{Očekivana
  vrijednost slijedi iz parnosti funkcije oko $x=\mu$, a varijansa se može
odrediti primjenom parcijalne integracije.} Ako je $\mu=0$ i $\sigma^2=1$, takva
raspodjela se zove \textit{standardna} normalna raspodjela. Na slici
\ref{fig:gauss-uni} je prikazan izgled normalnih raspodjela za nekoliko
vrijednosti parametara $\mu$ i $\sigma^2$.

\begin{figure}[h]
  \centering
  \fig{0.31}{gauss_uni_std}{$\mu=0,\ \sigma^2=1$}{}
  \fig{0.31}{gauss_uni_deltamean}{$\mu=-2,\ \sigma^2=1$}{}
  \fig{0.31}{gauss_uni_deltavar}{$\mu=-2,\ \sigma^2=4$}{}
	\caption{Primjeri normalnih raspodjela sa različitim parametrima}
  \label{fig:gauss-uni}
\end{figure}

\subsection{Funkcija raspodjele vjerovatnoće}

Funkcija raspodjele vjerovatnoće (\textit{CDF}) za slučajnu varijablu $X \sim
\mathcal{N}(0, 1)$ je data kao:

\begin{equation} \label{eq:gauss-cdf-std}
  P_X(x) = \Phi(x)
  = \frac{1}{\sqrt{2\pi}} \int_{-\infty}^{x} e^{-\frac{x^2}{2}} \ \mathrm dx
\end{equation}

Za integral u prethodnoj relaciji ne postoji zatvoren oblik, pa se iz tog
razloga CDF izražava preko nekih standardnih funkcija. Jedna takva funkcija je
upravo $\Phi$-funkcija definirana iznad.\footnote{Koriste se još i
  \textit{funkcija greške} $\mathrm{erf}(x)$ i $Q$-funkcija čije definicije se
  neznatno razlikuju.} Izražena preko ove funkcije, CDF slučajne varijable $X
  \sim \mathcal{N}(\mu, \sigma^2)$ postaje: \footnote{Ovo se jednostavno može
    dobiti odgovarajućom smjenom u integralu iz \ref{eq:gauss-cdf-std}.}

\begin{equation}
  P_X(x) = \Phi\left(\frac{x-\mu}{\sigma}\right)
\end{equation}

Normalna raspodjela posjeduje neke vrlo zanimljive i korisne osobine, koje je
najjednostavnije prikazati preko multivarijabilne normalne raspodjele.

\subsection{Multivarijabilna normalna raspodjela} \label{sec:gauss-multi}

Normalna raspodjela se može generalizirati i za vektore slučajnih varijabli na
sljedeći način.

\begin{definition}

  Za slučajni vektor $\bm X \in \mathbb{R}^k$ se kaže da je normalno
  raspodijeljen ako je njegova funkcija gustine raspodjele:
  %
  \begin{equation}
    p_{\bm X}(\bm x) 
    = \frac{1}{\sqrt{(2\pi)^k \det \bm C_{\bm X \bm X}}} e^{ -\frac{1}{2}(\bm x -
      \bm\mu_{\bm X})^\text T \bm C_{\bm X \bm X}^{-1} (\bm x - \bm\mu_{\bm X}) }
  \end{equation}
  Piše se $\bm X \sim \mathcal{N}(\bm\mu, \bm C_{\bm X\bm X})$.

\end{definition}

Vektor $\bm\mu$ je jednak očekivanju dok je matrica $\bm C_{\bm X\bm X}$
jednaka kovarijantnoj matrici slučajnog vekora $\bm X$.  Jedini uslov koji se
postavlja na parametre raspodjele jeste da matrica $\bm C_{\bm X\bm X}$ bude
ispravna kovarijantna matrica, tj. da bude simetrična i pozitivno definitna. 

\begin{theorem}
	\label{th:lin-tr-gauss}
	Neka je $\bm X \in \mathbb{R}^n$ normalno raspodijeljen slučajni vektor sa
	srednjom vrijednosti $\bm\mu_{\bm X}$ i kovarijantnom matricom $\bm C_{\bm
	X\bm X}$. Neka je $\bm A \in \mathbb{R}^{m\times n}$, $\bm b \in
	\mathbb{R}^n$, pri čemu je $m\le n$. Tada je vektor $\bm Y = \bm A \bm X + \bm
	b$ također normalno raspodijeljen sa srednjom vrijednosti $\bm\mu_{\bm Y} =
	\bm A\bm\mu_{\bm X} + \bm b$ i kovarijantnom matricom $\bm C_{\bm Y\bm Y} =
	\bm A \bm C_{\bm X\bm X} \bm A^\mathrm T$.

\end{theorem}

Jednostavnije rečeno, svaka linearna transformacija Gaussovog vektora uz
eventualnu translaciju, ponovo daje Gaussov vektor.

\begin{corollary}
	Svaki podvektor normalno raspodijeljenog slučajnog vektora $\bm X$ je također
	normalno raspodijeljen. Drugim riječima, sve marginalne raspodjele slučajnog
	vektora $\bm X$ su također normalne.
\end{corollary}

Ovo je lako pokazati. Naime, neki podvektor $\bm Y = [X_{i_1}\ X_{i_2}\ \cdots\
X_{i_m}]^\text T$ vektora $\bm X$ se može dobiti formiranjem matrice $\bm A$
koja u $j$ - tom redu sadrži jedinicu na poziciji $i_j$, dok su ostali elementi
jednaki nuli. Za vektor $\bm b$ se uzima nul-vektor.  Parametri raspodjele
novodobijenog vektora se jednostavno određuju primjenom teorema
\ref{th:lin-tr-gauss}. Ovaj princip će biti ilustrovan na vektoru dvije slučajne
varijable kroz sljedeći primjer.

\begin{exmp}
  Neka je data slučajni vektor $\bm X = \vecrow{X_1}{X_2}^\mathrm T$ sa
  raspodjelom $\mathcal{N}(\bm\mu, \bm \Sigma)$. Odrediti marginalne raspodjele
  za ovaj vektor.
\end{exmp}

Neka je:

\begin{eqnarray} \label{eq:kovarijansa-2d}
	\bm\mu = \left[\begin{array}{c}
		\mu_1 \\ \mu_2
	\end{array}\right],\ 
	\bm\Sigma = \left[\begin{array}{cc}
	  \sigma_1^2 & \rho\sigma_1\sigma_2 \\  \rho\sigma_1\sigma_2 & \sigma_2^2
	\end{array}\right]
\end{eqnarray}

Slučajne varijable $X_1$ i $X_2$ se mogu zapisati na sljedeći način:
\begin{eqnarray}
	X_1 = \vecrow{1}{0} \veccol{X_1}{X_2} = \bm A_1 \bm X,
	\\
	X_2 = \vecrow{0}{1} \veccol{X_1}{X_2} = \bm A_2 \bm X,
\end{eqnarray}

Primjenom teorema \ref{th:lin-tr-gauss} se zaključuje da će $X_1$ i
$X_2$ također biti iz normalne raspodjele:

\begin{align*}
	X_1 &\sim \mathcal{N}(\bm A_1\bm\mu, \bm A_1 \bm\Sigma \bm A_1^\mathrm T)
		= \mathcal{N}(\mu_1, \sigma_1^2) \\
	X_2 &\sim \mathcal{N}(\bm A_2\bm\mu, \bm A_2 \bm\Sigma \bm A_2^\mathrm T)
		= \mathcal{N}(\mu_2, \sigma_2^2)
\end{align*}

Posmatrajući zapis \eqref{eq:kovarijansa-2d} jasno je da je $\rho$ Pearsonov
korelacioni koeficijent koji predstavlja mjeru koreliranosti između slučajnih
varijabli slučajnog vektora $\bm X$.

Na slici \ref{fig:gauss2} su prikazane dvodimenzionalne normalne PDF za neke
karakteristične vrijednosti kovarijantne matrice. Također su prikazane i
marginalne PDF koje su skalirane po visini radi jasnijeg prikaza.

\begin{figure}[H]
  \centering
  \fig{0.31}{gauss_multi_std}{$\bm\mu=\bm 0,\ \bm C_{\bm X\bm X} = \bm I$}{fig:gauss2a}
  \fig{0.31}{gauss_multi_corr+}{$\rho = 0.6$}{fig:gauss2b}
  \fig{0.31}{gauss_multi_corr-}{$\rho = -0.6$}{fig:gauss2c}
  \fig{0.31}{gauss_multi_corr1}{$\sigma_1^2=1, \sigma_2^2=5, \rho \approx 1$}{fig:gauss2d}
  \fig{0.31}{gauss_multi_y}{$\sigma_1^2=1,\ \sigma_2^2=5, \rho=0$}{fig:gauss2e}
  \fig{0.31}{gauss_multi_xcorr-}{$\sigma_1^2=4, \sigma_2^2=1, \rho=-0.6$}{fig:gauss2f}
	\caption{Primjeri normalnih raspodjela sa različitim parametrima}
	\label{fig:gauss2}
\end{figure}

Na slici \ref{fig:gauss2a} je prikazana standardna normalna raspodjela sa nultom
srednjom vrijednosti i jediničnom kovarijantnom matricom. Raspodjela na slici
\ref{fig:gauss2b} je izmijenjena tako da su varijable negativno korelirane. Na
slici \ref{fig:gauss2c} je ova koreliranost ista po modulu, ali negativna. Na
slici \ref{fig:gauss2b} je prikazana normalna raspodjela pri čemu su varijable
maksimalno pozitivno korelirane. Na slici \ref{fig:gauss2e} je prikazana
normalna raspodjela nekoreliranog slučajnog vektora, kod kojeg je varijansa
slučajne varijable $X_2$ veća nego kod varijable $X_1$. Konačno, na slici
\ref{fig:gauss2f} je prikazana normalna raspodjela sa pozitivnom korelacijom,
pri čemu je varijansa varijable $X_1$ veća nego kod varijable $X_2$.\\

\noindent
Dolazi se do sljedećih zaključaka:
\begin{enumerate}
	\item Povećavanjem varijanse neke varijable dolazi do "razvlačenja" raspodjele
		po odgovarajućoj koordinati
	\item Povećanjem koreliranosti varijabli dolazi do koncentracije raspodjele
		oko pravca koji zavisi od kovarijantne matrice.
\end{enumerate}

\begin{corollary}
  Suma Gaussovih varijabli ponovo predstavlja Gaussovu varijablu. Ako su ove
  varijable nezavisne sa raspodjelama $\mathcal{N}(\mu_i, \sigma_i^2)$,
  $i=1,...,n$, njihova suma će imati raspodjelu
  $\mathcal{N}(\sum_{i=1}^{n}\mu_i, \sum_{i=1}^{n}\sigma_i^2)$.\footnote{
    Može se dokazati uzimanjem $\bm A = \left[
        \begin{array}{cccc}
          1 & 1 & \cdots & 1
        \end{array}\right]$
    i primjenom teorema \ref{th:lin-tr-gauss} na vektor od $n$ nezavisnih
    Gaussovih slučajnih varijabli.
  }
\end{corollary}

Ova osobina je povezana sa centralnim graničnim teoremom (poglavlje
\ref{sec:clt}). Za kraj, biće navedena jedna zanimljiva osobina
dvodimenzionalne normalne raspodjele, bez dokaza.

\begin{theorem}

  Slučajne varijable $X_1$, $X_2$ koje čine normalno raspodijeljen slučajni
  vektor $\bm X = (X_1, X_2)$ su nezavisne ako i samo ako su nekorelirane.

\end{theorem}

