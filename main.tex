\documentclass[11pt, a4paper]{article}

\usepackage[utf8]{inputenc}
\usepackage[T1]{fontenc}
\usepackage[bosnian]{babel} 
\usepackage{graphicx}
\usepackage{grffile}
\usepackage{longtable}
\usepackage{wrapfig}
\usepackage{rotating}
\usepackage[normalem]{ulem}
\usepackage{amsmath}
\usepackage{textcomp}
\usepackage{amssymb}
\usepackage[unicode]{hyperref}
\usepackage{bm}
\usepackage{indentfirst} 
\usepackage{caption} 
\usepackage{subcaption} 
\usepackage{float}
\usepackage{tabularx} 
\usepackage{listings}
\usepackage{xcolor} 
\usepackage{inconsolata}

\newtheorem{theorem}{Teorem}[section]
\newtheorem{corollary}{Posljedica}
\newtheorem{exmp}{Primjer}[section]
\newtheorem{definition}{Definicija}[section]
\newtheorem{property}{Osobina}

\counterwithin{figure}{section}
\counterwithin{equation}{section}

\captionsetup{font=small}

% PATH
\makeatletter
\def\input@path{{contents/}}
\makeatother
\graphicspath{{img/}{_build/img/}}

\newcommand{\fig}[4]{
  \begin{subfigure}{#1\textwidth}
    \includegraphics[width=\textwidth]{#2}
    \caption{#3}
    \label{#4}
    \vspace{5pt}
  \end{subfigure}
}
\newcommand{\pr}{\mathrm{Pr}}

\date{\today}
\title{\bfseries
  \huge Signali i sistemi 2 \\[24pt]
  {\LARGE
    Jednodimenzionalne i višedimenzionalne normalne raspodjele, zakon velikih
    brojeva i centralni granični teorem} \\[24pt]
  {\normalfont \Large Autor: Haris Gušić}
  \vfill
}

% LISTINGS SETUP
\definecolor{code}{HTML}{576ECC}
\definecolor{gray}{HTML}{FAFAFA}
\definecolor{grn}{rgb}{.208,.678,.45}
\definecolor{const}{HTML}{DB8E41}

\lstset{
  language=Python,
  breaklines=true,
  keywordstyle=\bfseries\color{code},
  basicstyle={\footnotesize\ttfamily},
  commentstyle=\color{grn}\ttfamily,
  stringstyle=\color{const}\ttfamily,
  tabsize=4,
  showstringspaces=false,
  backgroundcolor=\color{gray},
  frame=tblr,
  framextopmargin=4pt,
  framexbottommargin=4pt,
  upquote=true,
  columns=fullflexible,
}
\lstset{literate=%
   *{0}{{{\color{const}0}}}1
    {1}{{{\color{const}1}}}1
    {2}{{{\color{const}2}}}1
    {3}{{{\color{const}3}}}1
    {4}{{{\color{const}4}}}1
    {5}{{{\color{const}5}}}1
    {6}{{{\color{const}6}}}1
    {7}{{{\color{const}7}}}1
    {8}{{{\color{const}8}}}1
    {9}{{{\color{const}9}}}1
}

%%% BEGIN DOCUMENT %%%

\begin{document}

\pagenumbering{gobble}
\maketitle
\newpage
\tableofcontents
\newpage

\begin{abstract}

  U ovom radu su prezentovane tri izuzetno važne teme iz oblasti vjerovatnoće.
  Najprije je prikazan pojam \textit{normalne raspodjele}, te formalno uvedene
  neke od najvažnijih osobina ove raspodjele. Razmatran je slučaj raspodjele
  jedne varijable, te slučaj raspodjele više varijabli. U narednom poglavlju je
  detaljno intuitivno opisan \textit{zakon velikih brojeva}, koji pomaže u
  intuitivnom shvatanju primjenjivosti teorija vjerovatnoće. Osnovni korišteni
  matematički pojam za ovu svrhu je pojam \textit{slučajne sekvence}. Uvedene su
  formalne definicije i formulisane teoreme. U narednom poglavlju je prikazan
  \textit{centralni granični teorem}, počevši od probabilističkih osobina suma
  slučajnih varijabli, pri čemu razmatranje postepeno konvergira ka tvrđenju
  teorema.  Ponuđen je i dokaz teorema korištenjem \textit{karakterističnih
  funkcija}. Na kraju poglavlja je intuitivno pokazana veza između zakona
  velikih brojeva i centralnog graničnog teorema. Uz svako razmatranje su
  priložene grafičke ilustracije i odgovarajući primjeri primjene u praksi.

\end{abstract}

\newpage
\pagenumbering{arabic}

\section{Uvod} \label{sec:intro}

\indent
U ovom radu su detaljno prezentovane tri teme koje su izuzetno važne u oblasti
vjerovatnoće i statistike.  \\

\textit{Normalna raspodjela} je jedna od najčesće susretanih raspodjela vjerovatnoće u
jako velikom broju naučnih i inženjerskih oblasti, kao i u svakodnevnom životu.
U ovom radu su prezentovane normalne raspodjele za slučaj jedne, kao i za slučaj
više varijabli, uz detaljnu ilustraciju ovih funkcija za neke karakteristične
vrijednosti parametara. Ova raspodjela ima neke vrlo korisne osobine koje će
biti predstavljene u nastavku rada. Neke od tih osobina su direktno povezane sa
centralnim graničnim teoremom.

\textit{Zakon velikih brojeva} je značajan rezultat koji opravdava neke
intuicije vezane za sam pojam vjerovatnoće. Jedna takva intuicija jeste da će
relativna frekvencija nekog događaja u nizu eksperimenata biti približno jednaka
vjerovatnoći tog događaja. Druga je da će srednja vrijednost slučajne varijable
dobivena eksperimentom biti približno jednaka očekivanoj vrijednosti dobivenoj
na osnovu probabilističkog modela. Alternativno tumačenje je da zakon velikih
brojeva opravdava primjenjivost aksiomatske teorije vjerovatnoće u praksi. U
nastavku rada su navedene još neke korisne implikacije. Zakon velikih brojeva je
pogodno opisati korištenjem pojma slučajne sekvence (procesa), što će biti
urađeno u nastavku rada.

\textit{Centralni granični teorem} daje objašnjenje zašto se normalna raspodjela
susreće toliko često u svakodnevnom životu i u praksi. Ovaj teorem objašnjava
ponašanje vjerovatnoće sume velikog broja slučajnih varijabli, pod određenim
uslovima koji su dosta blagi sa aspekta praktičnih primjena. U najjednostavnijem
obliku teorem govori da će suma nezavisnih slučajnih varijabli imati približno
normalnu raspodjelu.  Zavisno od karaktera slučajnih varijabli, konvergencija ka
normalnoj raspodjeli može biti izuzetno brza. Fokus ovog razmatranja je na
jednostavnijim primjerima, i to za slučaj jedne varijable. Radi potpunosti su
navedene i neke generalnije formulacije teorema. \\


\section{Normalna raspodjela} \label{sec:gauss}
\newcommand*{\vecrow}[2]{\left[\begin{array}{cc}#1&#2\end{array}\right]}
\newcommand*{\veccol}[2]{\left[\begin{array}{c}#1\\#2\end{array}\right]}

U prirodi se najčešće susreće normalna raspodjela, koja je opisana funkcijom
gustine raspodjele vjerovatnoće:

$$p_X(x) = \mathcal{N}(\mu,\sigma^2)
	= \frac{1}{\sqrt{2\pi\sigma^2}} e^{-\frac{(x-\mu)^2}{2\sigma^2}}$$

\begin{figure}[h]
  \centering
  \begin{subfigure}[b]{0.28\textwidth}
    \centering
    \includegraphics[width=\textwidth]{gauss_uni_std.pdf}
    \caption{$\mu=0,\ \sigma^2=1$}
  \end{subfigure}
	\begin{subfigure}[b]{0.28\textwidth}
		\centering
    \includegraphics[width=\textwidth]{gauss_uni_deltamean.pdf}
    \caption{$\mu=-2,\ \sigma^2=1$}
	\end{subfigure}
	\begin{subfigure}[b]{0.28\textwidth}
		\centering
    \includegraphics[width=\textwidth]{img/gauss_uni_deltavar.pdf}
    \caption{$\mu=-2,\ \sigma^2=4$}
	\end{subfigure}
	\caption{Primjeri normalnih raspodjela sa različitim parametrima}
\end{figure}

\subsection{Multivarijabilna normalna raspodjela}

Normalna raspodjela se može generalizirati i za vektore slučajnih varijabli.
Za slučajni vektor $\bm X \in \mathbb{R}^k$ se kaže da je normalno
raspodijeljen ako je njegova funkcija gustine raspodjele:

$$p_{\bm X}(\bm x) = \mathcal{N}(\bm\mu,\bm C_{\bm X\bm X}) =
	\frac{1}{\sqrt{(2\pi)^k \det \bm C_{\bm X \bm X}}}
	e^{
		-\frac{1}{2}(\bm x - \bm\mu_{\bm X})^\text T
		\bm C_{\bm X \bm X}^{-1}
		(\bm x - \bm\mu_{\bm X})
 	 }$$

Jedini uslov koji se postavlja na parametre ove raspodjele jeste da matrica
$\bm C_{\bm X\bm X}$ bude ispravna kovarijantna matrica, tj. da bude simetrična
i pozitivno semi-definitna.

\begin{theorem}
	\label{th:lin-tr-gauss}
	Neka je $\bm X \in \mathbb{R}^n$ normalno raspodijeljen slučajni vektor sa
	srednjom vrijednosti $\bm\mu_{\bm X}$ i kovarijantnom matricom $\bm C_{\bm
	X\bm X}$. Neka je $\bm A \in \mathbb{R}^{m\times n}$, $\bm b \in
	\mathbb{R}^n$, pri čemu je $m\le n$. Tada je vektor $\bm Y = \bm A \bm X + \bm
	b$ također normalno raspodijeljen sa srednjom vrijednosti $\bm\mu_{\bm Y} =
	\bm A\bm\mu_{\bm X} + \bm b$ i kovarijantnom matricom $\bm C_{\bm Y\bm Y} =
	\bm A \bm C_{\bm X\bm X} \bm A^\mathrm T$.

\end{theorem}

Jednostavnije rečeno, svaka linearna transformacija Gaussovog vektora uz
eventualnu translaciju, ponovo daje Gaussov vektor.

\begin{corollary}
	Svaki podvektor normalno raspodijeljenog slučajnog vektora $\bm X$ je također
	normalno raspodijeljen. Drugim riječima, sve marginalne pdf slučajnog
	vektora $\bm X$ su također normalne raspodjele.
\end{corollary}

Ovo je lako pokazati. Naime, neki podvektor $\bm Y = [X_{i_1}\ X_{i_2}\ \cdots\
X_{i_m}]^\text T$ vektora $\bm X$ se može dobiti formiranjem matrice $\bm A$
koja u $j$ - tom redu sadrži jedinicu na poziciji $i_j$, dok su ostali elementi
jednaki nuli. Za vektor $\bm b$ se uzima nul-vektor.  Parametri raspodjele
novodobijenog vektora se jednostavno određuju primjenom teorema
\ref{th:lin-tr-gauss}. Ovaj princip će biti ilustrovan na vektoru dvije slučajne
varijable kroz sljedeći primjer.

\begin{exmp}
	Neka je data normalna raspodjela $p_{\bm X}(\bm x) = p_{X_1,X_2}(x_1, x_2) =
	\mathcal{N}(\bm\mu, \bm \Sigma)$. Odrediti marginalne pdf $p_{X_1}(x)$ i
	$p_{X_2}(x)$.
\end{exmp}

Neka je:

\begin{eqnarray} \label{eq:kovarijansa-2d}
	\bm\mu = \left[\begin{array}{c}
		\mu_1 \\ \mu_2
	\end{array}\right],\ 
	\bm\Sigma = \left[\begin{array}{cc}
	  \sigma_1^2 & \rho\sigma_1\sigma_2 \\  \rho\sigma_1\sigma_2 & \sigma_2^2
	\end{array}\right]
\end{eqnarray}

Slučajne varijable $X_1$ i $X_2$ se mogu zapisati na sljedeći način:
\begin{eqnarray}
	X_1 = \vecrow{1}{0} \veccol{X_1}{X_2} = \bm A_1 \bm X,
	\\
	X_2 = \vecrow{0}{1} \veccol{X_1}{X_2} = \bm A_2 \bm X,
\end{eqnarray}

Primjenom teoreme \ref{th:lin-tr-gauss} se zaključuje da će $p_{X_1}$ i
$p_{X_2}$ također biti normalne raspodjele:

\begin{align*}
	p_{X_1} &= \mathcal{N}(\bm A_1\bm\mu, \bm A_1 \bm\Sigma \bm A_1^\mathrm T)
		= \mathcal{N}(\mu_1, \sigma_1^2) \\
	p_{X_2} &= \mathcal{N}(\bm A_2\bm\mu, \bm A_2 \bm\Sigma \bm A_2^\mathrm T)
		= \mathcal{N}(\mu_2, \sigma_2^2)
\end{align*}

Posmatrajući zapis \eqref{eq:kovarijansa-2d} jasno je da je $\rho$ Pearsonov
korelacioni koeficijent koji predstavlja mjeru koreliranosti između slučajnih
varijabli slučajnog vektora $\bm X$.

Na slici \ref{fig:gauss2} su prikazane dvodimenzionalne normalne pdf za neke
karakteristične vrijednosti kovarijantne matrice. Također su prikazane i
marginalne pdf koje su skalirane po visini radi jasnijeg prikaza.

\begin{figure}[H]
  \centering
	\begin{subfigure}[b]{0.3\textwidth}
		\centering
		\includegraphics[width=\textwidth]{gauss_multi_std.pdf}
		\caption{$\bm\mu=\bm 0,\ \bm C_{\bm X\bm X} = \bm I$}
		\label{fig:gauss2a}
	\end{subfigure}
	\begin{subfigure}[b]{0.3\textwidth}
		\centering
    \includegraphics[width=\textwidth]{img/gauss_multi_corr+.pdf}
    \caption{$\rho = 0.6$}
		\label{fig:gauss2b}
	\end{subfigure}
	\begin{subfigure}[b]{0.3\textwidth}
		\centering
    \includegraphics[width=\textwidth]{img/gauss_multi_corr-.pdf}
    \caption{$\rho = -0.6$}
		\label{fig:gauss2c}
	\end{subfigure}
	\begin{subfigure}[b]{0.3\textwidth}
		\centering
    \includegraphics[width=\textwidth]{img/gauss_multi_corr1.pdf}
    \caption{$\sigma_1^2=1, \sigma_2^2=5, \rho \approx 1$}
		\label{fig:gauss2e}
	\end{subfigure}
	\begin{subfigure}[b]{0.3\textwidth}
		\centering
    \includegraphics[width=\textwidth]{gauss_multi_y.pdf}
    \caption{$\sigma_1^2=1,\ \sigma_2^2=5, \rho=0$}
		\label{fig:gauss2d}
	\end{subfigure}
	\begin{subfigure}[b]{0.3\textwidth}
		\centering
    \includegraphics[width=\textwidth]{gauss_multi_xcorr-.pdf}
    \caption{$\sigma_1^2=4, \sigma_2^2=1, \rho=0.6$}
		\label{fig:gauss2f}
	\end{subfigure}
	\caption{Primjeri normalnih raspodjela sa različitim parametrima}
	\label{fig:gauss2}
\end{figure}

Na slici \ref{fig:gauss2a} je prikazana standardna normalna raspodjela sa nultom
srednjom vrijednosti i jediničnom kovarijantnom matricom. Raspodjela na slici
\ref{fig:gauss2b} je izmijenjena tako da su varijable negativno korelirane. Na
slici \ref{fig:gauss2c} je ova koreliranost ista po modulu, ali negativna. Na
slici \ref{fig:gauss2b} je prikazana normalna raspodjela pri čemu su varijable
maksimalno pozitivno korelirane. Na slici \ref{fig:gauss2e} je prikazana
normalna raspodjela nekoreliranog slučajnog vektora, kod kojeg je varijansa
slučajne varijable $X_2$ veća nego kod varijable $X_1$. Konačno, na slici
\ref{fig:gauss2f} je prikazana normalna raspodjela sa pozitivnom korelacijom,
pri čemu je varijansa varijable $X_1$ veća nego kod varijable $X_2$.\\

\noindent
Dolazi se do sljedećih zaključaka:
\begin{enumerate}
	\item Povećavanjem varijanse neke varijable dolazi do "razvlačenja" raspodjele
		po odgovarajućoj koordinati
	\item Povećanjem koreliranosti varijabli dolazi do koncentracije raspodjele
		oko pravca koji zavisi od kovarijantne matrice.
\end{enumerate}

\begin{corollary}
  Suma Gauss-ovih varijabli ponovo predstavlja Gauss-ovu varijablu. Ako su ove
  varijable nezavisne sa pdf $\mathcal{N}(\mu_i, \sigma_i^2)$, $i=1,...,n$,
  njihova suma će imati raspodjelu $ \mathcal{N}(\sum_{i=1}^{n}\mu_i,
  \sum_{i=1}^{n}\sigma_i^2)$.\footnote{
    Može se dokazati uzimanjem $\bm A = \left[
        \begin{array}{cccc}
          1 & 1 & \cdots & 1
        \end{array}\right]$
    i primjenom teorema \ref{th:lin-tr-gauss} na vektor od $n$ nezavisnih
    Gauss-ovih slučajnih varijabli.
  }
\end{corollary}



\section{Zakoni velikih brojeva} \label{sec:lln}
\subsection{Konvergencija slučajnih sekvenci}

Ako su $X_i$, $i=1,2,...$ slučajne varijable, onda se sa $\{X_n\}$ označava
slučajna sekvenca (proces). Razlika između slučajne sekvence i determinističke
sekvence jeste u tome što vrijednost sekvence zavisi od ishoda nekog
eksperimenta.  Ako je $\Omega$ skup svih ishoda nekog eksperimenta, onda se
slučajna sekvenca može zapisati kao $\{X_n(\omega)\}$, pri čemu $\omega \in
\Omega$. Ovaj posljednji zapis će se često koristiti u ostatku poglavlja. Bitno
je primijetiti da sekvenca $X_n$ predstavlja determinističku sekvencu za
svaku konkretnu (poznatu) vrijednost $\omega$.

Kao u slučaju determinističkih sekvenci, od interesa je posmatrati ponašanje
slučajnih sekvenci (procesa) u beskonačnosti. Međutim, kod slučajnih sekvenci
postoji više različitih pojmova konvergencije. U ovom poglavlju će biti uvedene
neke od njih. 

\begin{definition}[Konvergencija svuda]
  Za slučajnu sekvencu $\{X_n\}$ se kaže da konvergira svuda (sigurno) ka
  slučajnoj varijabli $X$ ako je
  \begin{equation}
    \lim_{n\to\infty} X_n(\omega) = X(\omega)
  \end{equation}
  za svako $\omega \in \Omega$.
\end{definition}

% Primjer TODO rm
%\begin{exmp}
%  Neka je data slučajna sekvenca $X_n(\omega) = \omega^n$, pri čemu je
%  eksperiment opisan uniformnom funkcijom gustine raspodjele $p(\omega)$ na
%  intervalu $[0, 0.5]$, koji se podudara sa skupom ishoda $\Omega$. Ispitati
%  sigurnu konvergenciju sekvence $X_n$.
%\end{exmp}

%Jednostavno se zaključuje da je
%\begin{equation}
%  \lim_{n\to\infty} X_n(\omega)
%  = \lim_{n\to\infty} \omega^n = 0,\ \forall \omega \in \Omega
%\end{equation}

%Dakle, sekvenca $X_n(\omega)$ konvergira svuda ka $X=0$. U ovom
%primjeru sekvenca konvergira ka istom broju za svaku vrijednost $\omega$, tj.
%nezavisno od ishoda eksperimenta.

% Primjer
\begin{exmp}
  Neka je slučajna sekvenca $X_n(\omega) = \frac{\omega n}{n+1}$, $\omega \in
  \Omega = \mathbb{R}$. Ispitati sigurnu konvergenciju.
\end{exmp}

Vrijedi
\begin{equation}
  \lim_{n\to\infty} X_n(\omega) = \omega \lim_{n\to\infty}\frac{n}{n+1} = \omega
\end{equation}

Dakle, sekvenca $X_n(\omega)$ konvergira svuda ka slučajnoj varijabli $X(\omega)
= \omega$. U ovom primjeru slučajna sekvenca $X_n$ konvergira za svako $\omega$.
Međutim granična vrijednost zavisi od $\omega$, tj. od ishoda eksperimenta, što
je opisano slučajnom varijablom $X$.

\begin{exmp} \label{ex:converge-surely}
  Neka je $X_n(\omega) = \omega^n$, pri čemu je $\Omega= \mathbb{R}$ eksperiment
  opisan uniformnom PDF:
  $$p(\omega) = 1_{[0,1]}(\omega) = \begin{cases}
      1, \omega \in [0,1]\ \\ 0,\ \text{\normalfont inače}
    \end{cases}
  $$
  Ispitati sigurnu konvergenciju sekvence $X_n$.
\end{exmp}

Jasno je da sekvenca $\omega^n$ konvergira ka $X=0$ za $\omega \in [0,1)$, te ka
$X=1$ za $\omega = 1$. Međutim, za sve ostale vrijednosti $\omega$, sekvenca
divergira. Dakle, sekvenca $X_n$ ne konvergira svuda.\\

Međutim, konvergencija svuda u praksi često nije potrebna. Na primjer, nije od
velikog interesa da li će sekvenca $X_n$ iz prethodnog primjera konvergirati za
$\omega \in \mathbb{R}\setminus[0,1]$, jer je taj događaj nemoguć. Zato se uvodi
pojam konvergencije skoro svuda.

\begin{definition}[Konvergencija skoro svuda]
  Za slučajnu sekvencu $\{X_n\}$ se kaže da konvergira skoro svuda (skoro
  sigurno, ili sa vjerovatnoćom 1) ka slučajnoj varijabli $X$ ako je
  \begin{equation}
    \lim_{n\to\infty} X_n(\omega) = X(\omega)
  \end{equation}
  za svako $\omega$ iz nekog skupa $\Lambda$ za koji vrijedi
  $\pr(\omega\in\Lambda)=1$.
\end{definition}

Sekvenca iz primjera \ref{ex:converge-surely} konvergira skoro sigurno jer
sekvenca konvergira za $\forall \omega \in \Lambda = [0,1]$, pri čemu je
$\pr(\omega \in [0,1]) = 1$.

\begin{definition}[Konvergencija po vjerovatnoći]
  Za slučajnu sekvencu $\{X_n\}$ se kaže da konvergira po vjerovatnoći ka
  slučajnoj varijabli $X$ ako za svako fiksno $\varepsilon > 0$ vrijedi
  \footnote{
    Ovaj uslov se često (ekvivalentno) navodi kao
    $\lim_{n\to\infty} \pr(|X_n-X|>\varepsilon) = 0$
  }
  \begin{equation}
    \lim_{n\to\infty} \pr\left(|X_n - X| \le \varepsilon\right) = 1
  \end{equation}
\end{definition}

Grubo rečeno, sekvenca koja konvergira po vjerovatnoći je takva da relativno
veliki broj realizacija sekvence dovoljno dobro aproksimira slučajnu varijablu
$X$ u beskonačnosti. Ekvivalentno, to znači da relativno mali broj realizacija
nedovoljno dobro aproksimira slučajnu varijablu $X$ u beskonačnosti. Iz skoro
sigurne konvergencije slijedi konvergencija po vjerovatnoći, dok obrat ne važi.
Intuitivno, konvergencija po vjerovatnoći znači "Svaka realizacija
vjerovatno konvergira", dok konvergencija skoro svuda znači "Vjerovatno je da
svaka realizacija konvergira". \\

U nastavku će sa $P_X(x)$ biti označena funkcija raspodjele
vjerovatnoće slučajne varijable $X$.
\begin{definition}[Konvergencija po distribuciji]
  Za slučajnu sekvencu $\{X_n\}$ se kaže da konvergira po raspodjeli ka
  slučajnoj varijabli $X$ ako je
  \begin{equation}
    \lim_{n\to\infty} P_{X_n}(x) = P_X(x)
  \end{equation}
  u svakoj tački $x$ u kojoj je funkcija $P_X$ neprekidna.
\end{definition}

\begin{exmp}
  Neka je $\{X_n\}$ slučajna sekvenca, pri čemu svaki $X_n$ ima raspodjelu
  $\mathcal{N}(1/n, 1)$. Ispitati konvergenciju po distribuciji. TODO
\end{exmp}

Prije nego što se formuliše zakon velikih brojeva, potrebno je definirati
srednju vrijednost uzorka (engl. \textit{sample mean}).
\begin{definition}[Srednja vrijednost uzorka]

  Neka je $\{X_n\}$ slučajna sekvenca. Slučajna varijabla $S_n$
  definirana kao 
  \begin{equation} \label{eq:sample-mean}
    S_n = \frac{1}{n} \sum_{i=1}^{n} X_i
  \end{equation}
  se naziva srednja vrijednost uzorka.

\end{definition}
Od praktičnog interesa je odrediti ponašanje sekvence $\{S_n\}$ za velike
vrijednosti $n$. Preciznije, od interesa su sljedeće osobine:
Sljedeća pitanja su od praktičnog interesa:
\begin{itemize}
  \item Ponašanje empirijski dobivene srednje vrijednosti $S_n$, konvergencija i
    na koji način konvergencija zavisi od sekvence $\{X_n\}$.
  \item Statističke osobine malih varijacija sekvence $S_n$ oko srednje
    vrijednosti.
\end{itemize}

\subsection{Generalizacije i srodne tvdnje}



\section{Centralni granični teorem} \label{sec:clt}
U ovom poglavlju će biti objašnjen jedan od najznačajnijih rezultata u oblasti
vjerovatnoće. Centralni granični teorem opisuje ponašanje sume velikog broja
slučajnih varijabli. Prije nego što se formuliše teorem, biće napravljeno uvodno
razmatranje.

\subsection{Raspodjela sume slučajnih varijabli}

Neka su $K$ i $L$ dvije diskretne slučajne varijable sa masenim funkcijama
raspodjele (pmf) $p_K(k)$ i $p_L(l)$ respektivno. Neka je potrebno naći pmf
$p_M(m)$ slučajne varijable $M=K+L$.
%
\begin{align*}
  p_M(m) = \pr(M=m) = \pr(k\in\mathbb{Z}, l=m-k)
  = \sum_{k=-\infty}^{\infty} p_{K,L}(k,m-k)
\end{align*}

Ako su varijable $K$ i $L$ međusobno nezavisne, onda vrijedi $p_{K,L}(k,l) =
p_K(k)p_L(l)$, te se prethodni izraz može dodatno pojednostaviti:
%
\begin{align*}
  p_M(m) = \sum_{k=-\infty}^{\infty} p_K(k)p_L(m-k) = p_K * p_L (m)
\end{align*}

Dakle, suma nezavisnih diskretnih slučajnih varijabli će imati pmf koja je
jednaka diskretnoj konvoluciji pmf tih varijabli. Sličan zaključak vrijedi i za
slučaj kontinualnih varijabli.

\begin{theorem} %TODO da li ga nazvati teoremom?

  Ako su $X_1,X_2,...,X_n$ međusobno nezavisne slučajne varijable sa funkcijama
  gustine vjerovatnoće $p_{X_i}$, $i=1,...,n$ respektivno, onda njihova suma $S$
  ima funkciju gustine vjerovatnoće:
  $$p_S(s) = p_{X_1} * p_{X_2} * \cdots * p_{X_n} (s)$$

\end{theorem}

Prethodno razmatranje će sada biti intuitivno objašnjeno na primjeru.  Radi
jednostavnosti, razmatranje će biti ograničeno na diskretni slučaj.

\begin{exmp} \label{ex:novcic}
  
Neka su $X_i$ slučajne varijable koje uzimaju vrijednosti iz $\{0,1\}$ zavisno
od ishoda bacanja nepravednog novčića, pri čemu je $p_X(0)= p$, $p_X(1)=q=1-p$.
Potrebno je odrediti broj bacanja $K$ u kojima je $X_i=1$, ako je novčić bačen
$n$ puta.

\end{exmp}

Traženi broj $K$ također predstavlja slučajnu varijablu i može se izraziti
kao:\footnote{Pošto $X_i$ mogu uzeti samo vrijednosti 0 ili 1, njihova suma je
jednaka broju njih koji su jednaki 1}

$$K = \sum_{i=1}^{n} X_i$$
\\

Pošto su slučajne varijable $X_i$ IID, pmf od $K$ se dobija konvolucijom:
$$p_K(k) = \underbrace{p_X * p_X * \cdots * p_X}_{n\text{ puta}}(k)$$

Jednostavan način da se odredi ova konvolucija jeste korištenjem
$\mathcal{Z}$ transformacije.\footnote{\ $p_K(k)$ se može odrediti i čisto
kombinatornim pristupom.}
Naime, vrijedi: %TODO footnote binomna
\begin{equation}
  \mathcal{Z}\{p_K(k)\}
  = \left[\mathcal{Z}\{p_X(k)\}\right]^n
  = (p+qz^{-1})^n
  = \sum_{k=0}^{n} \left(\frac{n}{k}\right) p^k q^{n-k} z^{-k}
\end{equation}

Direktnom primjenom inverzne $ \mathcal{Z}$ transformacije se dobiva:
\begin{equation}
  p_K(k) = \left(\frac{n}{k}\right) p^kq^{n-k}
\end{equation}

Radi jednostavnosti, u nastavku će biti pretpostavljeno da je novčić pravedan,
tj. $p=q=1/2$, pa će odgovarajuća pmf biti:
\begin{equation} \label{eq:binom-pmf}
  p_K(k) = \frac{1}{2^n} \left(\frac{n}{k}\right)
\end{equation}

Na slikama \ref{fig:binom} je prikazan izgled pmf iz \eqref{eq:binom-pmf} za
razne vrijednosti $n$.  Na slici \ref{fig:binom:a} je prikazana pmf $p_K(k)$, za
jedno bacanje, koja se naravno podudara sa $p_X(k)$. Postoje dva moguća ishoda,
sa jednakom vjerovatnoćom koja iznosi 0.5. Ako se novčić baci drugi put (slika
\ref{fig:binom:b}), mogući ishodi za $K$ su $0, 1, 2$, pri čemu je ishod $1$
dvostruko vjerovatniji od ostalih, jer se može dobiti na dva načina
($1=0+1=1+0$), dok se ostali mogu dobiti na samo jedan način ($0=0+0$, $2=1+1$).
Ako se novčić baci treći put (slika \ref{fig:binom:c}), mogući ishodi su
$0,1,2,3$, pri čemu vrijednosti 1 i 2 imaju najveću vjerovatnoću. Na slikama
\ref{fig:binom:d} i \ref{fig:binom:e} su prikazane pmf za $n=8$ i $n=50$
respektivno. Već na slici \ref{fig:binom:d} se može primijetiti da je pmf
skoncentrisana oko sredine intervala, dok se na slici \ref{fig:binom:e} vidi da
pmf aproksimira Gaussovu funkciju. Ova pojava se može formalno iskazati
\textit{De Moivre-Laplace}-ovim teoremom koji tvrdi da je:

\begin{equation}
  \left(\frac{n}{k}\right) p^kq^{n-k} \simeq \frac{1}{\sqrt{2\pi npq}}
  e^{-\frac{(k-np)^2}{2npq}}
\end{equation}
Dakle, binomna raspodjela aproksimira Gaussovu funkciju za veliko $n$.
Prethodni primjer je dosta regularan zbog činjenice da je novčić pravedan, pa su
sve pmf simetrične oko neke centralne tačke. U nastavku su navedeni neki bitni
zaključci, koji vrijede u općem slučaju. Ovi zaključci direktno slijede iz
osobina konvolucije, a također su i dosta intuitivni.
TODO
%\begin{enumerate}
  %\item Suma dvije varijable uzima 
%\end{enumerate}

\begin{figure}[H]
  \centering
  \begin{subfigure}[b]{0.3\textwidth}
    \centering
    \includegraphics[width=\textwidth]{clt_binom_1.pdf}
    \caption{$n=1$}
    \label{fig:binom:a}
  \end{subfigure}
  \vspace{10pt}
	\begin{subfigure}[b]{0.3\textwidth}
		\centering
    \includegraphics[width=\textwidth]{clt_binom_2.pdf}
    \caption{$n=2$}
    \label{fig:binom:b}
	\end{subfigure}
	\begin{subfigure}[b]{0.3\textwidth}
		\centering
    \includegraphics[width=\textwidth]{clt_binom_3.pdf}
    \caption{$n=3$}
    \label{fig:binom:c}
	\end{subfigure}
	\begin{subfigure}[b]{0.3\textwidth}
		\centering
    \includegraphics[width=\textwidth]{clt_binom_8.pdf}
    \caption{$n=8$}
    \label{fig:binom:d}
	\end{subfigure}
	\begin{subfigure}[b]{0.3\textwidth}
		\centering
    \includegraphics[width=\textwidth]{clt_binom_50.pdf}
    \caption{$n=50$}
    \label{fig:binom:e}
	\end{subfigure}
	\caption{Binomna raspodjela i De Moivre-Laplace aproksimacija}
  \label{fig:binom}
\end{figure}

\subsection{Formulacija centralnog graničnog teorema}

Postoje različite formulacije centralnog graničnog teorema (engl.
\textit{central limit theorem} - CLT), ovdje će biti data jedna od
najjednostavnijih, mada ne i najopćenitija.
% TODO u uvodu dati prevod CLT
\begin{theorem} \label{th:clt}
  Neka je $\{X_i, i=1,2,...\}$ sekvenca IID slučajnih varijabli sa
  konačnim očekivanjem $\mu$ i konačnom varijansom $\sigma^2$. Tada
  sekvenca
  \begin{equation}
    S_n = \frac{1}{\sqrt{n}} \sum_{i=1}^{n} \frac{X_i-\mu}{\sigma}
  \end{equation}
  konvergira po distribuciji ka normalnoj raspodjeli $\mathcal{N}(0, 1)$.
\end{theorem}

Nažalost, način kako je definirana sekvenca $S_n$ nije najbolji oblik za
intuitivno shvatanje teorema. Prirodnije bi bilo sekvencu $S_n$ definirati kao
$S_n=\sum_{i=1}^{n} X_i$, bez suvišnih oznaka. Međutim, ukoliko se koristi takva
definicija, nastaje problem u tome što takva sekvenca neće konvergirati po
distribuciji. Raspodjela takve sekvence će sa povećanjem $n$ bolje i bolje
aproksimirati \textbf{neku} Gaussovu raspodjelu $\mathcal{N}(n\mu,n\sigma^2)$
koja zavisi od $n$, ali očigledno ovakva sekvenca raspodjela neće težiti ka
istoj raspodjeli. Još jedan prirodan način definiranja sekvence $S_n$ bi bio kao
kao srednja vrijednost uzorka sekvence $X_n$. Međutim ovdje bi problem napravio
zakon velikih brojeva uslijed kojeg će sa povećanjem $n$ varijansa težiti u
nulu. Ovaj slučaj će biti dodatno razmotren u poglavlju \ref{sec:dokaz-clt}.
Radi dodatne ilustracije, u slučaju definiranja $S_n = \left(\sum_{i=1}^n
X_i\right)/\sqrt{n}$, varijansa bi konvergirala, ali bi očekivanje $\mu
\sqrt{n}$ divergiralo.  Međutim, uprkos ovih problema, često se susreću
formulacije  sekvencu $S_n$ na prethodno navedene načine, no takve formulacije
su manje precizne. \\
%TODO Provjeriti ovo predzadnje

Centralni granični teorem ima nekoliko interesantnih posljedica koje su
objašnjene u nastavku.

\setcounter{corollary}{0}
\begin{corollary}

  Suma velikog broja IID slučajnih varijabli sa očekivanjem $\mu$ i varijansom
  $\sigma^2$ imati raspodjelu koja aproksimira $\mathcal{N}(n\mu, n\sigma^2)$,
  nezavisno od raspodjele varijabli koje učestvuju u sumi. 

\end{corollary}

Ovakva interpretacija teorema je jednostavnija i intuitivnija nego sama
formulacija teorema koja je data iznad.

%TODO ref to eqn

%TODO teorem ne pomaže intuiciji

\begin{corollary}
  Konvolucijom pozitivne funkcije sa samom sobom mnogo puta dobiva se približno
  Gaussova funkcija.
\end{corollary}

Ovo je već ilustrovano u primjeru \ref{ex:novcic}, mada je tamo bila razmatrana
raspodjela u diskretnom slučaju koja je dosta regularna i simetrična. U nastavku
će isti princip biti prikazan na nekoliko različitih kontinualnih funkcija. Neka
je $f(x)$ neka nenegativna funkcija dovoljno "lijepih" osobina, i neka je:

\begin{equation}
  g_n(x) = f^{*n}(x) := \underbrace{f*f\cdots*f}_{n\text{ puta}}\ (x)
\end{equation}

Također, neka je $\widetilde{g}_n(x)$ aproksimacija funkcije $g_n(x)$ Gaussovom
funkcijom.

Na slici \ref{fig:convolution} su prikazane tri funkcije: četvrtka, zašumljena
četvrtka i neka treća funkcija, zajedno sa odgovarajućim funkcijama $g_n(x)$ za
dvije različite vrijednosti broja $n$ za svaku od te tri funkcije. Za slučaj
četvrtke (slike \ref{fig:rect} - \ref{fig:rect2}), vidi se da funkcionalni niz
$g_n(x)$ jako brzo konvergira ka Gaussovoj funkciji. Već za $n=1$ aproksimacija
je dosta dobra, dok za $n=2$ funkcija $g_n(x)$ jako liči na Gaussovu. Ponašanje
zašumljene četvrtke (slike \ref{fig:noise} - \ref{fig:noise2}) se može opisati
na sličan način. U slučaju treće funkcije (slike \ref{fig:exp} -
\ref{fig:exp2}), konvergencija je dosta sporija. Na primjer za $n=1$, ne vidi se
nikakva sličnost funkcije $g_n(x)$ sa Gaussovom.  Povećanjem broja $n$,
postepeno se uočava sličnost. Međutim, čak i za $n=9$ aproksimacija je lošija
nego u slučaju četvrtke za $n=2$.
%TODO praksa pokazuje za nesimetrične...

\begin{figure}[H]
  \centering
  \begin{tabularx}{\textwidth}{ccc}
    \begin{tabular}{c}
      \fig{0.27}{clt_conv_rect}{Četvrtka}{fig:rect} \\[20pt]
      \fig{0.27}{clt_conv_rect_1}{$n=1$}{fig:rect1} \\
      \fig{0.27}{clt_conv_rect_2}{$n=2$}{fig:rect2}
    \end{tabular}
    &
    \begin{tabular}{c}
      \fig{0.27}{clt_conv_noise}{Zašumljena četvrtka}{fig:noise} \\[20pt]
      \fig{0.27}{clt_conv_noise_1}{$n=1$}{fig:noise1} \\[20pt]
      \fig{0.27}{clt_conv_noise_2}{$n=2$}{fig:noise2}
    \end{tabular}
    &
    \begin{tabular}{c}
      \fig{0.27}{clt_conv_exp}{Treća funkcija}{fig:exp} \\
      \fig{0.27}{clt_conv_exp_1}{$n=1$}{fig:exp1} \\
      \fig{0.27}{clt_conv_exp_2}{$n=9$}{fig:exp2}
    \end{tabular}
  \end{tabularx}
	\caption{Primjeri kontinualnih funkcija i njihovih konvolucija sa samim sobom}
  \label{fig:convolution}
\end{figure}

\subsection{Dokaz centralnog graničnog teorema} \label{sec:dokaz-clt}

U ovom poglavlju će biti ponuđen dokaz centralnog graničnog teorema. Za potrebe
ovog dokaza potrebno je definirati pojam karakteristične funkcije.

\begin{definition}[Karakteristična funkcija]
  Neka je $X$ slučajna varijabla. Funkcija 
  \begin{equation}
    \varphi_X(u) := E\left[e^{juX}\right]
    = \int_{-\infty}^{\infty} p_X(x) e^{jux} \ \mathrm dx
  \end{equation}
  se naziva karakterističnom funkcijom slučajne varijable $X$.
\end{definition}

Iz definicije se odmah vidi da je karakteristična funkcija
$\varphi_X(u)$\footnote{U literaturi se koriste još i oznake $M_X$ i $\Phi_X$}
jednaka Fourierovoj transformaciji funkcije $p_X(x)$.

\begin{property}
  Ako su $X_i$, $i=1,2,...,n$ nezavisne slučajne varijable i
  $S=X_1+X_2+\cdots+X_n$ njihova suma, onda vrijedi:
  \begin{equation}
    \varphi_S(u) = \prod_{i=1}^{n} \varphi_{X_i}(u)
  \end{equation}
  Specijalno, ako se radi o IID varijablama, onda je
  \begin{equation} \label{eq:char-power}
    \varphi_S(u) = [\varphi_X(u)]^n
  \end{equation}
\end{property}

Dokaz:

\begin{align}
  \varphi_S(u) = E\left[e^{ju(X_1+\cdots+X_n)}\right]
  &= \idotsint\limits_{\mathbb{R}^n}
    p_{X_1,...,X_n}(x_1,...,x_n) e^{ju(X_1+\cdots+X_n)} \ \text d\bm x \nonumber \\
  &= \int_{-\infty}^{\infty} p_{X_1}(x_1)e^{juX_1}\ \mathrm dx_1 \cdots
    \int_{-\infty}^{\infty} p_{X_n}(x_n)e^{juX_n}\ \mathrm dx_n
\end{align}

\begin{property}[Normalna raspodjela]
  Slučajna varijabla iz normalne raspodjele $X \sim \mathcal{N}(\mu, \sigma^2)$
  ima karakterističnu funkciju: %TODO footnote UPUTA?
  \begin{equation} \label{eq:char-norm}
    \varphi_X(u) = e^{j\mu u} e^{-\frac{u^2}{2\sigma^2}}
  \end{equation}
\end{property}

Nakon što su uvedene potrebne definicije i osobine, slijedi dokaz centralnog
graničnog teorema, koji nije potpuno matematski strog, jer su neki detalji
izostavljeni radi jednostavnijeg razumijevanja. \\

\subsection*{Dokaz CLT}

Neka je $\{X_i, i=1,2,...\}$ sekvenca IID slučajnih varijabli sa očekivanjem
$\mu$ i standardnom devijacijom $\sigma$ i neka je
\begin{equation}
  S_n := \frac{1}{\sqrt{n}} \sum_{i=1}^{n} \frac{X_i-\mu}{\sigma}
\end{equation}
Neka se definira pomoćna sekvenca $Y_i := \frac{X_i-\mu}{\sigma}$, za koju
očigledno vrijedi $E[Y_i] = 0$, $\text{Var}[Y_i] = 1$. Ako slučajne varijable
$X_i$ imaju karakteristične funkcije $\varphi_{X_i}(u) = \varphi_X(u)$, onda
slučajne varijable imaju karakteristične funkcije:
\begin{equation} \label{eq:phi_y}
  \varphi_{Y_i}(u) = E\left[e^{ju\frac{X_i-\mu}{\sigma}}\right]
  = \int_{-\infty}^{\infty} p_X(x)e^{ju\frac{x-\mu}{\sigma}} \ \mathrm dx
  = e^{-ju\frac{\mu}{\sigma}} \varphi_X\left(\frac{u}{\sigma}\right)
  = \varphi_Y(u)
\end{equation}
Pogodno je uvesti još jednu pomoćnu sekvencu $Z_{i,n} := Y_i/\sqrt{n}$. Na
sličan način kao u \eqref{eq:phi_y} može se pokazati da je:
\begin{equation}
  \varphi_{Z_{i,n}}(u)
  = \varphi_{Z,n}(u) := \varphi_Y\left(\frac{u}{\sqrt{n}}\right)
\end{equation}

Pod određenim uslovima koji ovdje neće biti navedeni, funkcija $\varphi_{Z,n}$
se može aproksimirati Taylorovim razvojem:
\begin{equation} \label{eq:taylor}
  \varphi_{Z,n}(u) = \varphi_Y\left(\frac{u}{\sqrt{n}}\right) 
  = \varphi_Y(0) + \varphi_Y^{'}(0)\frac{u}{\sqrt{n}}
  + \varphi_Y^{''}(0)\frac{u^2}{2n} + o\left(\frac{u^2}{2n}\right),
    \text{ kad } n\to\infty
\end{equation}

Dalje vrijedi:
\begin{align*}
  \varphi_Y(0)
    &= \left.\int_{-\infty}^{\infty} p_{Y_i}(y) e^{juy} \ \text dy\right|_{u=0}
    = \int_{-\infty}^{\infty} p_{Y_i}(y) \ \text dy = 1,\  \forall i \\
  \varphi_Y^{'}(0)
    &= \left.\int_{-\infty}^{\infty} jyp_{Y_i}(y)e^{juy} \ \text dy \right|_{u=0}
    = jE[Y_i] = 0,\ \forall i \\
  \varphi_Y^{''}(0)
    &= \left.-\int_{-\infty}^{\infty} y^2 p_{Y_i}(y)e^{juy} \ \text dy\right|_{u=0}
    = -E[Y_i^2] = -1,\ \forall i
\end{align*}

Formula \eqref{eq:taylor} sada postaje:
\begin{equation}
  \varphi_{Z,n}(u) = 1 - \frac{u^2}{2n} + o\left(\frac{u^2}{2n}\right)
\end{equation}

Pošto vrijedi:
\begin{equation}
  S_n = \sum_{i=1}^{n} Z_{i,n}
\end{equation}
na osnovu formule \eqref{eq:char-power} ova sekvenca će imati karakterističnu
funkciju:
\begin{equation}
  \varphi_{S_n}(u) = \left[\varphi_{Z,n}(u)\right]^n
\end{equation}
U graničnom procesu:
\begin{equation} \label{eq:clt-limit}
  \lim_{n\to\infty} \varphi_{S_n}(u)
  = \lim_{n\to\infty} [\varphi_{Z,n}(u)]^n
  = \lim_{n\to\infty}
    \left(1 - \frac{u^2}{2n} + o\left(\frac{u^2}{2n}\right)\right)^n
    = e^{-\frac{u^2}{2}}
\end{equation}
Na osnovu \eqref{eq:char-norm}, raspodjela koja ima karakterističnu funkciju kao
u \eqref{eq:clt-limit} je upravo normalna $\mathcal{N}(0,1)$. $\blacksquare$ \\

Zanimljivo je razmotriti šta se dešava kada je sekvenca $S_n$ definirana kao
srednja vrijednost uzorka iz \eqref{eq:sample-mean}. Ako se definira pomoćna
sekvenca $W_i = \frac{X_i}{n}$, tada je

\begin{equation}
  S_n = \frac{1}{n}\sum_{i=1}^{\infty} X_i = \sum_{i=1}^{n} W_{i,n}
\end{equation}

Jednostavnim manipulacijama se dolazi do:
\begin{equation}
  W_{i,n} = \frac{X_i}{n} = \frac{\sigma \sqrt{n} Z_{i,n}+\mu}{n}
  = \frac{\sigma}{\sqrt{n}} Z_{i,n} + \frac{\mu}{n}
\end{equation}
gdje su $Y_i$, $Z_{i,n}$ definirane na isti način kao u dokazu CLT.

Primjenom definicije karakteristične funkcije se dobiva:
\begin{equation}
  \varphi_{W_{i,n}}(u) = \varphi_{W,n}(u)
  = e^{j\frac{\mu}{n} u} \varphi_{Z,n}\left(\frac{\sigma u}{\sqrt{n}}\right)
\end{equation}

Dalje vrijedi:

\begin{equation}
  \varphi_{S_n}(u) = [\varphi_{W,n}(u)]^n
  = e^{j\mu u} \left[\varphi_{Z,n}\left(\frac{\sigma u}{\sqrt{n}}\right)\right]^n
\end{equation}

Iskorištavanjem Taylorovog razvoja iz \eqref{eq:taylor}, dolazi se do:
\begin{equation}
  \lim_{n\to\infty}
    \left[\varphi_{Z,n}\left(\frac{\sigma u}{\sqrt{n}}\right)\right]^n
    = \lim_{n\to\infty} \left[1-\frac{\sigma^2u^2}{2n^2} +
      o\left(\frac{\sigma^2u^2}{2n^2}\right)\right]^n
    = 1
\end{equation}
Konačno,
\begin{equation}
  \lim_{n\to\infty} \varphi_{S_n}(u) = e^{ju\mu}
\end{equation}
Primjenom inverzne Fourierove transformacije se dobiva da je
\begin{equation}
  \lim_{n\to\infty} p_{S_n}(s) = \delta(s-\mu)
\end{equation}

Ovaj zaključak se intuitivno podudara sa tvrđenjem zakona velikih brojeva.
Naime, što je uzorak veći, srednja vrijednost uzorka se manje rasipa oko srednje
vrijednosti probabilističkog modela. Za uzorak beskonačne veličine, srednja
vrijednost uzorka se podudara sa matematičkim očekivanjem. \textbf{Drugim
riječima, što je veći uzorak, nesigurnost u tačnost očekivanja
procijenjenog na osnovu tog uzorka je manja.} %TODO
% ovo prebaciti u LLN?

\subsection{Praktične primjene}

\subsection{Generalizacije i srodne tvrdnje}



\addcontentsline{toc}{section}{Literatura}
\nocite{*}
\bibliographystyle{IEEEtranETF}
\bibliography{literature}

\section*{Prilog}
\addcontentsline{toc}{section}{Prilog}

U prilogu su dati svi programski kodovi koji su korišteni za simulacije i
grafičke vizualizacije u radu. Svi programi su napisani u programskom jeziku
\textit{Python}, verzija 3.8.

\appendix
\newcommand{\lst}[1]{
  \begin{figure}[H]
    \centering
    \begingroup \fontencoding{T1}\selectfont
      \lstinputlisting{#1}
    \endgroup
    %\caption{Uključenje potrebnih biblioteka}
  \end{figure}
}

\section{Zajednički programski kodovi za sve simulacije}
%
Programski kodovi u nastavku koriste sljedeće biblioteke i funkcije.
\lst{_build/src/shared.py_1}
\lst{_build/src/shared.py_2}
\lst{_build/src/shared.py_3}
\lst{_build/src/shared.py_4}
\lst{_build/src/shared.py_5}

\section{Programski kodovi za demonstraciju normalne raspodjele}
%
Programski kodovi u nastavku koriste sljedeće biblioteke i funkcije:
\lst{_build/src/gauss.py_1}
\lst{_build/src/gauss.py_2}
\lst{_build/src/gauss.py_3}
\lst{_build/src/gauss.py_4}
\lst{_build/src/gauss.py_5}

\section{Programski kodovi za demonstraciju zakona velikih brojeva}
%
Programski kodovi u nastavku koriste sljedeće biblioteke funkcije.
\lst{_build/src/lln.py_1}
\lst{_build/src/lln.py_2}
\lst{_build/src/lln.py_3}
\lst{_build/src/lln.py_4}
\lst{_build/src/lln.py_5}
\lst{_build/src/lln.py_6}

\section{Programski kodovi za demonstraciju centralnog graničnog teorema}
%
Programski kodovi u nastavku koriste sljedeće biblioteke i funkcije.
\lst{_build/src/clt.py_1}
\lst{_build/src/clt.py_2}
\lst{_build/src/clt.py_3}
\lst{_build/src/clt.py_4}
\lst{_build/src/clt.py_5}


\end{document}
