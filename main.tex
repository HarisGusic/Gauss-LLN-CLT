\documentclass[11pt]{article}

\usepackage[utf8]{inputenc}
\usepackage[T1]{fontenc}
\usepackage[bosnian]{babel} 
\usepackage{graphicx}
\usepackage{grffile}
\usepackage{longtable}
\usepackage{wrapfig}
\usepackage{rotating}
\usepackage[normalem]{ulem}
\usepackage{amsmath}
\usepackage{textcomp}
\usepackage{amssymb}
\usepackage[unicode]{hyperref}
\usepackage{bm}
\usepackage{indentfirst} 
\usepackage{caption} 
\usepackage{subcaption} 
\usepackage{float}
\usepackage{tabularx} 
\usepackage{listings}
\usepackage{xcolor} 
\usepackage{inconsolata}

\newtheorem{theorem}{Teorem}[section]
\newtheorem{corollary}{Posljedica}
\newtheorem{exmp}{Primjer}[section]
\newtheorem{definition}{Definicija}[section]
\newtheorem{property}{Osobina}

\counterwithin{figure}{section}
\counterwithin{equation}{section}

\captionsetup{font=small}

% PATH
\makeatletter
\def\input@path{{contents/}}
\makeatother
\graphicspath{{img/}{_build/img/}}

\newcommand{\fig}[4]{
  \begin{subfigure}{#1\textwidth}
    \includegraphics[width=\textwidth]{#2}
    \caption{#3}
    \label{#4}
    \vspace{5pt}
  \end{subfigure}
}
\newcommand{\pr}{\mathrm{Pr}}

\date{\today}
\title{Jednodimenzionalne i višedimenzionalne normalne raspodjele, zakon velikih
brojeva i centralni granični teorem}

% LISTINGS SETUP
\definecolor{code}{HTML}{576ECC}
\definecolor{gray}{HTML}{FAFAFA}
\definecolor{grn}{rgb}{.208,.678,.45}
\definecolor{const}{HTML}{DB8E41}

%\lstset{
%  language=Python,
%  breaklines=true,
%  keywordstyle=\bfseries\color{code},
%  basicstyle={\footnotesize\ttfamily},
%  commentstyle=\color{grn}\ttfamily,
%  stringstyle=\color{const}\ttfamily,
%  tabsize=4,
%  showstringspaces=false,
%  backgroundcolor=\color{gray},
%  %xleftmargin=0pt,
%  %xrightmargin=30pt,
%  frame=tblr,
%  %framexleftmargin=6pt,
%  %framexrightmargin=6pt,
%  framextopmargin=4pt,
%  framexbottommargin=4pt,
%  upquote=true,
%  columns=fullflexible,
%}
%\lstset{literate=%
%   *{0}{{{\color{const}0}}}1
%    {1}{{{\color{const}1}}}1
%    {2}{{{\color{const}2}}}1
%    {3}{{{\color{const}3}}}1
%    {4}{{{\color{const}4}}}1
%    {5}{{{\color{const}5}}}1
%    {6}{{{\color{const}6}}}1
%    {7}{{{\color{const}7}}}1
%    {8}{{{\color{const}8}}}1
%    {9}{{{\color{const}9}}}1
%}

%%% BEGIN DOCUMENT %%%

\begin{document}

\pagenumbering{gobble}
\maketitle
\newpage
\tableofcontents
\newpage

\abstract{

  U ovom radu su prezentovane tri izuzetno važne teme iz oblasti vjerovatnoće.
  Najprije je prikazan pojam \textit{normalne raspodjele}, te formalno uvedene
  neke od najvažnijih osobina ove raspodjele. Razmatran je slučaj raspodjele
  jedne varijable, te slučaj raspodjele više varijabli. U narednom poglavlju je
  detaljno intuitivno opisan \textit{zakon velikih brojeva}, koji pomaže u
  intuitivnom shvatanju primjenjivosti teorija vjerovatnoće. Osnovni korišteni
  matematički pojam za ovu svrhu je pojam \textit{slučajne sekvence}. Uvedene su
  formalne definicije i formulisane teoreme. U narednom poglavlju je prikazan
  \textit{centralni granični teorem}, počevši od probabilističkih osobina suma
  slučajnih varijabli, pri čemu razmatranje postepeno konvergira ka tvrđenju
  teorema.  Ponuđen je i dokaz teorema korištenjem \textit{karakterističnih
  funkcija}. Na kraju poglavlja je intuitivno pokazana veza između zakona
  velikih brojeva i centralnog graničnog teorema. Uz svako razmatranje su
  priložene grafičke ilustracije i odgovarajući primjeri primjene u praksi.

}

\newpage
\pagenumbering{arabic}

\section{Uvod} \label{sec:intro}

\indent
U ovom radu su detaljno prezentovane tri teme koje su izuzetno važne u oblasti
vjerovatnoće i statistike.  \\

\textit{Normalna raspodjela} je jedna od najčesće susretanih raspodjela vjerovatnoće u
jako velikom broju naučnih i inženjerskih oblasti, kao i u svakodnevnom životu.
U ovom radu su prezentovane normalne raspodjele za slučaj jedne, kao i za slučaj
više varijabli, uz detaljnu ilustraciju ovih funkcija za neke karakteristične
vrijednosti parametara. Ova raspodjela ima neke vrlo korisne osobine koje će
biti predstavljene u nastavku rada. Neke od tih osobina su direktno povezane sa
centralnim graničnim teoremom.

\textit{Zakon velikih brojeva} je značajan rezultat koji opravdava neke
intuicije vezane za sam pojam vjerovatnoće. Jedna takva intuicija jeste da će
relativna frekvencija nekog događaja u nizu eksperimenata biti približno jednaka
vjerovatnoći tog događaja. Druga je da će srednja vrijednost slučajne varijable
dobivena eksperimentom biti približno jednaka očekivanoj vrijednosti dobivenoj
na osnovu probabilističkog modela. Alternativno tumačenje je da zakon velikih
brojeva opravdava primjenjivost aksiomatske teorije vjerovatnoće u praksi. U
nastavku rada su navedene još neke korisne implikacije. Zakon velikih brojeva je
pogodno opisati korištenjem pojma slučajne sekvence (procesa), što će biti
urađeno u nastavku rada.

\textit{Centralni granični teorem} daje objašnjenje zašto se normalna raspodjela
susreće toliko često u svakodnevnom životu i u praksi. Ovaj teorem objašnjava
ponašanje vjerovatnoće sume velikog broja slučajnih varijabli, pod određenim
uslovima koji su dosta blagi sa aspekta praktičnih primjena. U najjednostavnijem
obliku teorem govori da će suma nezavisnih slučajnih varijabli imati približno
normalnu raspodjelu.  Zavisno od karaktera slučajnih varijabli, konvergencija ka
normalnoj raspodjeli može biti izuzetno brza. Fokus ovog razmatranja je na
jednostavnijim primjerima, i to za slučaj jedne varijable. Radi potpunosti su
navedene i neke generalnije formulacije teorema. \\


\section{Normalna raspodjela} \label{sec:gauss}

U prirodi se najčešće susreće normalna raspodjela, koja je opisana funkcijom
gustine raspodjele vjerovatnoće:

$$p_X(x) = \mathcal{N}(\mu,\sigma^2)
	= \frac{1}{\sqrt{2\pi\sigma^2}} e^{-\frac{(x-\mu)^2}{2\sigma^2}}$$

Slučajna varijabla je normalno raspodijeljena ako ima funkciju gustine
raspodjele vjerovatnoće (pdf) oblika:

\subsection{Multivarijabilna normalna raspodjela}

Normalna raspodjela se može generalizirati i za vektore slučajnih varijabli.
Za slučajni vektor $\bm X \in \mathbb{R}^k$ se kaže da je normalno
raspodijeljen ako je njegova funkcija gustine raspodjele:

$$p_{\bm X}(\bm x) = \mathcal{N}(\bm\mu,\bm C_{\bm X\bm X}) =
	\frac{1}{\sqrt{(2\pi)^k \det \bm C_{\bm X \bm X}}}
	e^{
		-\frac{1}{2}(\bm x - \bm\mu_{\bm X})^\text T
		\bm C_{\bm X \bm X}^{-1}
		(\bm x - \bm\mu_{\bm X})
 	 }$$

Jedini uslov koji se postavlja na parametre ove raspodjele jeste da matrica
$\bm C_{\bm X\bm X}$ bude ispravna kovarijantna matrica, tj. da bude simetrična
i pozitivno semi-definitna.

\begin{theorem}
	\label{th:lin-tr-gauss}
	Neka je $\bm X \in \mathbb{R}^n$ normalno raspodijeljen slučajni vektor sa
	srednjom vrijednosti $\bm\mu_{\bm X}$ i kovarijantnom matricom $\bm C_{\bm
	X\bm X}$. Neka je $\bm A \in \mathbb{R}^{m\times n}$, $\bm b \in
	\mathbb{R}^n$, pri čemu je $m\le n$. Tada je vektor $\bm Y = \bm A \bm X + \bm
	b$ također normalno raspodijeljen sa srednjom vrijednosti $\bm\mu_{\bm Y} =
	\bm A\bm\mu_{\bm X} + \bm b$ i kovarijantnom matricom $\bm C_{\bm Y\bm Y} =
	\bm A \bm C_{\bm X\bm X} \bm A^\mathrm T$.

\end{theorem}

Jednostavnije rečeno, svaka linearna transformacija Gaussovog vektora uz
eventualnu translaciju, ponovo daje Gaussov vektor.

\begin{corollary}
	Svaki podvektor normalno raspodijeljenog slučajnog vektora $\bm X$ je također
	normalno raspodijeljen. Drugim riječima, sve marginalne pdf slučajnog
	vektora $\bm X$ su također normalne raspodjele.
\end{corollary}

Ovo je lako pokazati. Naime, neki podvektor $\bm Y = [X_{i_1}\ X_{i_2}\
	 \cdots\ X_{i_m}]^\text T$ vektora $\bm X$ se može dobiti formiranjem matrice
$\bm A$ koja u $j$ - tom redu sadrži jedinicu na poziciji $i_j$, dok su
ostali elementi jednaki nuli. Za vektor $\bm b$ se uzima nul-vektor.
Parametri raspodjele novodobijenog vektora se jednostavno određuju primjenom
teorema \ref{th:lin-tr-gauss}.

\begin{exmp}
	Neka je data normalna raspodjela $p_{\bm X}(\bm x) = p_{X_1,X_2}(x_1, x_2) =
	\mathcal{N}(\bm\mu, \bm \Sigma)$. Odrediti marginalne pdf $p_{X_1}(x)$ i
	$p_{X_2}(x)$.
\end{exmp}

Neka je:

\begin{eqnarray}
	\bm\mu = \left[\begin{array}{c}
		\mu_1 \\ \mu_2
	\end{array}\right],\ 
	\bm\Sigma = \left[\begin{array}{cc}
	  \sigma_1^2 & \sigma_{12} \\ \sigma_{12} & \sigma_2^2
	\end{array}\right]
\end{eqnarray}

Slučajne varijable $X_1$ i $X_2$ se mogu zapisati na sljedeći način:
\newcommand*{\vecrow}[2]{\left[\begin{array}{cc}#1&#2\end{array}\right]}
\newcommand*{\veccol}[2]{\left[\begin{array}{c}#1\\#2\end{array}\right]}
\begin{eqnarray}
	X_1 = \vecrow{1}{0} \veccol{X_1}{X_2} = \bm A_1 \bm X,
	\\
	X_2 = \vecrow{0}{1} \veccol{X_1}{X_2} = \bm A_2 \bm X,
\end{eqnarray}

Primjenom teoreme \ref{th:lin-tr-gauss} se zaključuje da će $p_{X_1}$ i
$p_{X_2}$ također biti normalne raspodjele:

\begin{align*}
	p_{X_1} &= \mathcal{N}(\bm A_1\bm\mu, \bm A_1 \bm\Sigma \bm A_1^\mathrm T)
		= \mathcal{N}(\mu_1, \sigma_1^2) \\
	p_{X_2} &= \mathcal{N}(\bm A_2\bm\mu, \bm A_2 \bm\Sigma \bm A_2^\mathrm T)
		= \mathcal{N}(\mu_2, \sigma_2^2)
\end{align*}



\section{Zakoni velikih brojeva} \label{sec:lln}
Kao u slučaju determinističkih sekvenci, od interesa je posmatrati ponašanje
slučajnih sekvenci (procesa) u beskonačnosti. Međutim, kod slučajnih sekvenci
postoji više različitih pojmova konvergencije.

\begin{definition}[Konvergencija svuda]
  Za slučajnu sekvencu $\{X_n\}$ se kaže da konvergira svuda (sigurno) ka
  slučajnoj varijabli $X$ ako je
  \begin{equation}
    \lim_{n\to\infty} X_n(\omega) = X(\omega)
  \end{equation}
  za svako $\omega \in \Omega$.
\end{definition}

\begin{definition}[Konvergencija skoro svuda]
  Za slučajnu sekvencu $\{X_n\}$ se kaže da konvergira skoro svuda (skoro
  sigurno) ka slučajnoj varijabli $X$ ako je
  \begin{equation}
    \lim_{n\to\infty} X_n(\omega) = X(\omega)
  \end{equation}
  za svako $\omega$ iz nekog skupa $\Lambda$ za koji vrijedi $P(\Lambda)=1$.
\end{definition}

\begin{definition}[Konvergencija po vjerovatnoći]
  Za slučajnu sekvencu $\{X_n\}$ se kaže da konvergira po vjerovatnoći ka
  slučajnoj varijabli $X$ ako za
  svako fiksno $\varepsilon > 0$ vrijedi
  \begin{equation}
    \lim_{n\to\infty} P\left(|X_n - X| \le \varepsilon\right)
  \end{equation}
\end{definition}

TODTOTOTOTOTOTOTOTOTOTODDDDDDDDDDO

\begin{equation} \label{eq:sample-mean}
  S_n = \frac{1}{n} \sum_{i=1}^{n} X_i
\end{equation}




\section{Centralni granični teorem} \label{sec:clt}
U ovom poglavlju će biti objašnjen jedan od najznačajnijih rezultata u oblasti
vjerovatnoće. Centralni granični teorem opisuje ponašanje sume velikog broja
slučajnih varijabli. Prije nego što se formuliše centralni granični teorem, biće
napravljeno uvodno razmatranje.

\subsection{Raspodjela sume slučajnih varijabli}

Neka su $K$ i $L$ dvije diskretne slučajne varijable sa masenim funkcijama
raspodjele (pmf) $p_K(k)$ i $p_L(l)$ respektivno. Neka je potrebno naći pmf
$p_M(m)$ slučajne varijable $M=K+L$.
%
\newcommand{\pr}{\text{Pr}}
\begin{align*}
  p_M(m) = \pr(M=m) = \pr(k\in\mathbb{Z}, l=m-k)
  = \sum_{k=-\infty}^{\infty} p_{K,L}(k,m-k)
\end{align*}

Ako su varijable $K$ i $L$ međusobno nezavisne, onda vrijedi $p_{K,L}(k,l) =
p_K(k)p_L(l)$, te se prethodni izraz može dodatno pojednostaviti:
%
\begin{align*}
  p_M(m) = \sum_{k=-\infty}^{\infty} p_K(k)p_L(m-k) = p_K * p_L (m)
\end{align*}

Dakle, suma nezavisnih diskretnih slučajnih varijabli će imati pmf koja je
jednaka diskretnoj konvoluciji pmf tih varijabli. Sličan zaključak vrijedi i za
slučaj kontinualnih varijabli.

\begin{theorem} %TODO da li ga nazvati teoremom?

  Ako su $X_1,X_2,...,X_n$ međusobno nezavisne slučajne varijable sa funkcijama
  gustine vjerovatnoće $p_{X_i}$, $i=1,...,n$ respektivno, onda njihova suma $S$
  ima funkciju gustine vjerovatnoće:
  $$p_S(s) = p_{X_1} * p_{X_2} * \cdots * p_{X_n} (s)$$

\end{theorem}

Prethodno razmatranje će sada biti intuitivno objašnjeno na primjeru.  Radi
jednostavnosti, razmatranje će biti ograničeno na diskretni slučaj.

Neka su $X_i$ slučajne varijable koje uzimaju vrijednosti iz $\{0,1\}$ sa
jednakom vjerovatnoćom, zavisno od ishoda bacanja pravednog novčića. Neka se
takav novčić baca veliki broj puta $n$, pri čemu je od interesa broj $K$ koji
predstavlja broj takvih bacanja u kojima je $X_i=1$. $K$ također predstavlja
slučajnu varijablu i može se izraziti kao: %TODO footnote?
%Na taj način je dobivena slučajna sekvenca $\{X_i,\ i=1,2,...\}$.  TODO neke
%stvari prebaciti u LLN

%\begin{equation} %TODO drop
%  S_n = \frac{1}{n} \sum_{i=1}^n X_i
%\end{equation}

$$K = \sum_{i=1}^{n} X_i$$
\\

Pošto su slučajne varijable $X_i$ IID, pmf od $K$ se dobija konvolucijom:
$$p_K(k) = \underbrace{p_X * p_X * \cdots * p_X}_{n\text{ puta}}(k)$$

Jednostavan način da se odredi ova konvolucija jeste korištenjem
$\mathcal{Z}$ transformacije.\footnote{\ $p_K(k)$ se može odrediti i čisto
kombinatornim pristupom.}
Naime, vrijedi: %TODO footnote binomna
\begin{equation}
  \mathcal{Z}\{p_K(k)\}
  = \left[\mathcal{Z}\{p_X(k)\}\right]^n
  = (p+qz^{-1})^n
  = \sum_{k=0}^{n} \left(\frac{n}{k}\right) p^k q^{n-k} z^{-k}
\end{equation}

Direktnom primjenom inverzne $ \mathcal{Z}$ transformacije se dobiva:
$$p_K(k) = \left(\frac{n}{k}\right) p^kq^{n-k}$$

\begin{equation} \label{eq:binom-pmf}
  p_K(k) = \frac{1}{2^n} \left(\frac{n}{k}\right)
\end{equation}

Na slikama \ref{fig:binom} je prikazan izgled pmf iz \eqref{eq:binom-pmf} za
razne vrijednosti $n$.

Na slici \ref{fig:binom:a} je prikazana pmf $p_K(k)$, za jedno bacanje, koja se
naravno podudara sa $p_X(k)$. Postoje dva moguća ishoda, sa jednakom
vjerovatnoćom koja iznosi 0.5. Ako se novčić baci drugi put (slika
\ref{fig:binom:b}), mogući ishodi za $K$ su $0, 1, 2$, pri čemu je ishod $1$
dvostruko vjerovatniji od ostalih, jer se može dobiti na dva načina
($1=0+1=1+0$), dok se ostali mogu dobiti na samo jedan način ($0=0+0$, $2=1+1$).
Ako se novčić baci treći put (slika \ref{fig:binom:c}), mogući ishodi su
$0,1,2,3$, pri čemu vrijednosti 1 i 2 imaju najveću vjerovatnoću. Na slikama
\ref{fig:binom:d} i \ref{fig:binom:e} su prikazane pmf za $n=8$ i $n=50$
respektivno. Već na slici \ref{fig:binom:d} se može primijetiti da je pmf
skoncentrisana oko sredine intervala, dok se na slici \ref{fig:binom:e} vidi da
pmf aproksimira Gauss-ovu funkciju. Ova pojava se može formalno iskazati
\textit{De-Moivre-Laplace}-ovim teoremom koji tvrdi da je:
\begin{equation}
  \left(\frac{n}{k}\right) p^kq^{n-k} \simeq \frac{1}{\sqrt{2\pi npq}}
  e^{-\frac{(k-np)^2}{2npq}}
\end{equation}
Dakle, binomna raspodjela aproksimira Gauss-ovu funkciju za veliko $n$.
Prethodni primjer je dosta regularan zbog činjenice da je novčić pravedan, pa su
sve pmf simetrične oko neke centralne tačke. U nastavku su navedeni neki bitni
zaključci, koji vrijede u općem slučaju. Ovi zaključci direktno slijede iz
osobina konvolucije, a također su i dosta intuitivni.
TODO
%\begin{enumerate}
  %\item Suma dvije varijable uzima 
%\end{enumerate}



\begin{figure}[H]
  \centering
  \begin{subfigure}[b]{0.3\textwidth}
    \centering
    \includegraphics[width=\textwidth]{clt_binom_1.pdf}
    \caption{$n=1$}
    \label{fig:binom:a}
  \end{subfigure}
  \vspace{10pt}
	\begin{subfigure}[b]{0.3\textwidth}
		\centering
    \includegraphics[width=\textwidth]{clt_binom_2.pdf}
    \caption{$n=2$}
    \label{fig:binom:b}
	\end{subfigure}
	\begin{subfigure}[b]{0.3\textwidth}
		\centering
    \includegraphics[width=\textwidth]{clt_binom_3.pdf}
    \caption{$n=3$}
    \label{fig:binom:c}
	\end{subfigure}
	\begin{subfigure}[b]{0.3\textwidth}
		\centering
    \includegraphics[width=\textwidth]{clt_binom_8.pdf}
    \caption{$n=8$}
    \label{fig:binom:d}
	\end{subfigure}
	\begin{subfigure}[b]{0.3\textwidth}
		\centering
    \includegraphics[width=\textwidth]{clt_binom_50.pdf}
    \caption{$n=50$}
    \label{fig:binom:e}
	\end{subfigure}
	\caption{Binomna raspodjela i De Moivre-Laplace aproksimacija}
  \label{fig:binom}
\end{figure}

\subsection{Formulacija centralnog graničnog teorema}

\begin{theorem}
  Neka je 
\end{theorem}



\addcontentsline{toc}{section}{Literatura}
\nocite{*}
\bibliographystyle{IEEEtranETF}
\bibliography{literature}

\section*{Prilog}

\appendix

\section{Programski kodovi za demonstraciju normalne raspodjele}
\input{appendix-gauss}

\newcommand{\lst}[1]{
  \begin{figure}[H]
    \centering
    \begingroup \fontencoding{T1}\selectfont
      \lstinputlisting{#1}
    \endgroup
  \end{figure}
}
\noindent
\lst{_build/src/gauss.py_1}
\lst{_build/src/gauss.py_2}
\lst{_build/src/gauss.py_3}
\lst{_build/src/gauss.py_4}
\lst{_build/src/gauss.py_5}
\lst{_build/src/gauss.py_6}

\end{document}
