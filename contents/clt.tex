U ovom poglavlju će biti objašnjen jedan od najznačajnijih rezultata u oblasti
vjerovatnoće. Centralni granični teorem opisuje ponašanje sume velikog broja
slučajnih varijabli. Prije nego što se formuliše centralni granični teorem, biće
napravljeno uvodno razmatranje.

\subsection{Raspodjela sume slučajnih varijabli}

Neka su $K$ i $L$ dvije diskretne slučajne varijable sa masenim funkcijama
raspodjele (pmf) $p_K(k)$ i $p_L(l)$ respektivno. Neka je potrebno naći pmf
$p_M(m)$ slučajne varijable $M=K+L$.
%
\newcommand{\pr}{\text{Pr}}
\begin{align*}
  p_M(m) = \pr(M=m) = \pr(k\in\mathbb{Z}, l=m-k)
  = \sum_{k=-\infty}^{\infty} p_{K,L}(k,m-k)
\end{align*}

Ako su varijable $K$ i $L$ međusobno nezavisne, onda vrijedi $p_{K,L}(k,l) =
p_K(k)p_L(l)$, te se prethodni izraz može dodatno pojednostaviti:
%
\begin{align*}
  p_M(m) = \sum_{k=-\infty}^{\infty} p_K(k)p_L(m-k) = p_K * p_L (m)
\end{align*}

Dakle, suma nezavisnih diskretnih slučajnih varijabli će imati pmf koja je
jednaka diskretnoj konvoluciji pmf tih varijabli. Sličan zaključak vrijedi i za
slučaj kontinualnih varijabli.

\begin{theorem} %TODO da li ga nazvati teoremom?

  Ako su $X_1,X_2,...,X_n$ međusobno nezavisne slučajne varijable sa funkcijama
  gustine vjerovatnoće $p_{X_i}$, $i=1,...,n$ respektivno, onda njihova suma $S$
  ima funkciju gustine vjerovatnoće:
  $$p_S(s) = p_{X_1} * p_{X_2} * \cdots * p_{X_n} (s)$$

\end{theorem}

Prethodno razmatranje će sada biti intuitivno objašnjeno na primjeru.  Radi
jednostavnosti, razmatranje će biti ograničeno na diskretni slučaj.

\begin{exmp} \label{ex:novcic}
  
Neka su $X_i$ slučajne varijable koje uzimaju vrijednosti iz $\{0,1\}$ zavisno
od ishoda bacanja nepravednog novčića, pri čemu je $p_X(0)= p$, $p_X(1)=q=1-p$.
Potrebno je odrediti broj bacanja $K$ u kojima je $X_i=1$, ako je novčić bačen
$n$ puta.

\end{exmp}

Traženi broj $K$ također predstavlja slučajnu varijablu i može se izraziti
kao:\footnote{Pošto $X_i$ mogu uzeti samo vrijednosti 0 ili 1, njihova suma je
jednaka broju njih koji su jednaki 1}

$$K = \sum_{i=1}^{n} X_i$$
\\

Pošto su slučajne varijable $X_i$ IID, pmf od $K$ se dobija konvolucijom:
$$p_K(k) = \underbrace{p_X * p_X * \cdots * p_X}_{n\text{ puta}}(k)$$

Jednostavan način da se odredi ova konvolucija jeste korištenjem
$\mathcal{Z}$ transformacije.\footnote{\ $p_K(k)$ se može odrediti i čisto
kombinatornim pristupom.}
Naime, vrijedi: %TODO footnote binomna
\begin{equation}
  \mathcal{Z}\{p_K(k)\}
  = \left[\mathcal{Z}\{p_X(k)\}\right]^n
  = (p+qz^{-1})^n
  = \sum_{k=0}^{n} \left(\frac{n}{k}\right) p^k q^{n-k} z^{-k}
\end{equation}

Direktnom primjenom inverzne $ \mathcal{Z}$ transformacije se dobiva:
$$p_K(k) = \left(\frac{n}{k}\right) p^kq^{n-k}$$

\begin{equation} \label{eq:binom-pmf}
  p_K(k) = \frac{1}{2^n} \left(\frac{n}{k}\right)
\end{equation}

Na slikama \ref{fig:binom} je prikazan izgled pmf iz \eqref{eq:binom-pmf} za
razne vrijednosti $n$.

Na slici \ref{fig:binom:a} je prikazana pmf $p_K(k)$, za jedno bacanje, koja se
naravno podudara sa $p_X(k)$. Postoje dva moguća ishoda, sa jednakom
vjerovatnoćom koja iznosi 0.5. Ako se novčić baci drugi put (slika
\ref{fig:binom:b}), mogući ishodi za $K$ su $0, 1, 2$, pri čemu je ishod $1$
dvostruko vjerovatniji od ostalih, jer se može dobiti na dva načina
($1=0+1=1+0$), dok se ostali mogu dobiti na samo jedan način ($0=0+0$, $2=1+1$).
Ako se novčić baci treći put (slika \ref{fig:binom:c}), mogući ishodi su
$0,1,2,3$, pri čemu vrijednosti 1 i 2 imaju najveću vjerovatnoću. Na slikama
\ref{fig:binom:d} i \ref{fig:binom:e} su prikazane pmf za $n=8$ i $n=50$
respektivno. Već na slici \ref{fig:binom:d} se može primijetiti da je pmf
skoncentrisana oko sredine intervala, dok se na slici \ref{fig:binom:e} vidi da
pmf aproksimira Gauss-ovu funkciju. Ova pojava se može formalno iskazati
\textit{De-Moivre-Laplace}-ovim teoremom koji tvrdi da je:
\begin{equation}
  \left(\frac{n}{k}\right) p^kq^{n-k} \simeq \frac{1}{\sqrt{2\pi npq}}
  e^{-\frac{(k-np)^2}{2npq}}
\end{equation}
Dakle, binomna raspodjela aproksimira Gauss-ovu funkciju za veliko $n$.
Prethodni primjer je dosta regularan zbog činjenice da je novčić pravedan, pa su
sve pmf simetrične oko neke centralne tačke. U nastavku su navedeni neki bitni
zaključci, koji vrijede u općem slučaju. Ovi zaključci direktno slijede iz
osobina konvolucije, a također su i dosta intuitivni.
TODO
%\begin{enumerate}
  %\item Suma dvije varijable uzima 
%\end{enumerate}

\begin{figure}[H]
  \centering
  \begin{subfigure}[b]{0.3\textwidth}
    \centering
    \includegraphics[width=\textwidth]{clt_binom_1.pdf}
    \caption{$n=1$}
    \label{fig:binom:a}
  \end{subfigure}
  \vspace{10pt}
	\begin{subfigure}[b]{0.3\textwidth}
		\centering
    \includegraphics[width=\textwidth]{clt_binom_2.pdf}
    \caption{$n=2$}
    \label{fig:binom:b}
	\end{subfigure}
	\begin{subfigure}[b]{0.3\textwidth}
		\centering
    \includegraphics[width=\textwidth]{clt_binom_3.pdf}
    \caption{$n=3$}
    \label{fig:binom:c}
	\end{subfigure}
	\begin{subfigure}[b]{0.3\textwidth}
		\centering
    \includegraphics[width=\textwidth]{clt_binom_8.pdf}
    \caption{$n=8$}
    \label{fig:binom:d}
	\end{subfigure}
	\begin{subfigure}[b]{0.3\textwidth}
		\centering
    \includegraphics[width=\textwidth]{clt_binom_50.pdf}
    \caption{$n=50$}
    \label{fig:binom:e}
	\end{subfigure}
	\caption{Binomna raspodjela i De Moivre-Laplace aproksimacija}
  \label{fig:binom}
\end{figure}

\subsection{Formulacija centralnog graničnog teorema}

Postoje različite formulacije centralnog graničnog teorema (CLT), ovdje će biti
data najjednostavnija.

\begin{theorem} \label{th:clt}
  Neka je $\{X_i, i=1,2,...\}$ sekvenca IID slučajnih varijabli sa
  konačnom srednjom vrijednosti $\mu$ i konačnom varijansom $\sigma^2$. Tada
  sekvenca $$S_n = \sum_{i=1}^{n} X_i$$ konvergira po distribuciji ka normalnoj
  raspodjeli $\mathcal{N}(n\mu, n\sigma^2)$.
\end{theorem}

Teorem \ref{th:clt} govori da će suma velikog broja IID slučajnih varijabli
imati približno normalnu raspodjelu, nezavisno od raspodjele varijabli koje
učestvuju u sumi.

Često se u literaturi prethodni teorem formuliše tako što se sekvenca $S_n$
normalizuje na sljedeći način:

$$S_n = \frac{1}{\sqrt{n}} \sum_{i=1}^{n} \frac{X_i-\mu}{\sigma}$$
Ovako je postignuto da $E[S_n] = 0$, $Var[S_n] = 1$, pa će konvergencija biti
ka \textit{standardnoj normalnoj raspodjeli} $\mathcal{N}(0,1)$.
Međutim, ovakva formulacija je ekvivalentna prethodnoj.

\begin{corollary}
  Konvolucijom pozitivne funkcije sa samom sobom mnogo puta dobiva se približno
  Gauss-ova funkcija.
\end{corollary}
Ovo je već ilustrovano u primjeru \ref{ex:novcic}, mada je tamo bila razmatrana
raspodjela koja je dosta regularna i simetrična. U nastavku će isti princip biti
prikazan na nekoliko različitih kontinualnih funkcija.

