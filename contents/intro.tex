
\indent
U ovom radu su detaljno prezentovane tri teme koje su izuzetno važne u oblasti
vjerovatnoće i statistike.  \\

\textit{Normalna raspodjela} je jedna od najčesće susretanih raspodjela vjerovatnoće u
jako velikom broju naučnih i inženjerskih oblasti, kao i u svakodnevnom životu.
U ovom radu su prezentovane normalne raspodjele za slučaj jedne, kao i za slučaj
više varijabli, uz detaljnu ilustraciju ovih funkcija za neke karakteristične
vrijednosti parametara. Navedene su i neke od najinteresantnijih osobina
normalne raspodjele, koje će se provlačiti kroz čitav rad.

\textit{Zakon velikih brojeva} je značajan rezultat koji opravdava neke
intuicije vezane za sam pojam vjerovatnoće. Jedna takva intuicija jeste da će
relativna frekvencija nekog događaja u nizu eksperimenata biti približno jednaka
vjerovatnoći tog događaja. Druga je da će srednja vrijednost slučajne varijable
dobivena eksperimentom biti približno jednaka očekivanoj vrijednosti dobivenoj
na osnovu probabilističkog modela. Alternativno tumačenje je da zakon velikih
brojeva opravdava primjenjivost aksiomatske teorije vjerovatnoće u praksi. U
nastavku rada su navedene još neke korisne implikacije. Također je detaljno
predstavljen pojam slučajne sekvence (procesa) i najvažniji pojmovi
konvergencije istih, čije razumijevanje je važno za razumijevanje materije. 

\textit{Centralni granični teorem} daje objašnjenje zašto se normalna raspodjela
susreće toliko često u svakodnevnom životu i u praksi. Ovaj teorem objašnjava
ponašanje vjerovatnoće sume velikog broja slučajnih varijabli, pod određenim
uslovima koji su dosta blagi sa aspekta praktičnih primjena. U najjednostavnijem
obliku govori da će suma nezavisnih slučajnih varijabli imati približno normalnu
raspodjelu.  Zavisno od karaktera slučajnih varijabli, konvergencija ka
normalnoj raspodjeli može biti izuzetno brza. Fokus ovog razmatranja je na
jednostavnijim primjerima, i to za slučaj jedne varijable. Radi potpunosti su
navedene i neke generalnije formulacije teorema. \\

Rad je koncipiran tako da većina razmatranja započinje predstavljanjem
intuitivnog uvoda (primjera), nakon čega se postepeno uvodi teoretska podloga na
način koji predstavlja razuman kompromis između matematske preciznosti i
jednostavnosti.  Neke tvrdnje su navedene bez dokaza, dok su neke značajnije
tvrdnje dokazane.  Definicije i tvrdnje su prikazane na matematski strog način.
Neki dokazi nisu potpuno matematski precizni, ali je to naglašeno na
odgovarajućim mjestima.  Osnovni fokus je da se čitalac upozna sa materijom na
način koji je najpogodniji za dalje istraživanje na tu temu, kao i razumijevanje
relevantne literature. 

