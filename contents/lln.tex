\subsection{Konvergencija slučajnih sekvenci}

Ako su $X_i$, $i=1,2,...$ slučajne varijable, onda se sa $\{X_n\}$ označava
slučajna sekvenca (proces). Razlika između slučajne sekvence i determinističke
sekvence jeste u tome što vrijednost sekvence zavisi od ishoda nekog
eksperimenta.  Ako je $\Omega$ skup svih ishoda nekog eksperimenta, onda se
slučajna sekvenca može zapisati kao $\{X_n(\omega)\}$, pri čemu $\omega \in
\Omega$. Ovaj posljednji zapis će se često koristiti u ostatku poglavlja. Bitno
je primijetiti da sekvenca $X_n$ predstavlja determinističku sekvencu za
svaku konkretnu (poznatu) vrijednost $\omega$.

Kao u slučaju determinističkih sekvenci, od interesa je posmatrati ponašanje
slučajnih sekvenci (procesa) u beskonačnosti. Međutim, kod slučajnih sekvenci
postoji više različitih pojmova konvergencije. U ovom poglavlju će biti uvedene
neke od njih. 

\begin{definition}[Konvergencija svuda]
  Za slučajnu sekvencu $\{X_n\}$ se kaže da konvergira svuda (sigurno) ka
  slučajnoj varijabli $X$ ako je
  \begin{equation}
    \lim_{n\to\infty} X_n(\omega) = X(\omega)
  \end{equation}
  za svako $\omega \in \Omega$.
\end{definition}

% Primjer TODO rm
%\begin{exmp}
%  Neka je data slučajna sekvenca $X_n(\omega) = \omega^n$, pri čemu je
%  eksperiment opisan uniformnom funkcijom gustine raspodjele $p(\omega)$ na
%  intervalu $[0, 0.5]$, koji se podudara sa skupom ishoda $\Omega$. Ispitati
%  sigurnu konvergenciju sekvence $X_n$.
%\end{exmp}

%Jednostavno se zaključuje da je
%\begin{equation}
%  \lim_{n\to\infty} X_n(\omega)
%  = \lim_{n\to\infty} \omega^n = 0,\ \forall \omega \in \Omega
%\end{equation}

%Dakle, sekvenca $X_n(\omega)$ konvergira svuda ka $X=0$. U ovom
%primjeru sekvenca konvergira ka istom broju za svaku vrijednost $\omega$, tj.
%nezavisno od ishoda eksperimenta.

% Primjer
\begin{exmp}
  Neka je slučajna sekvenca $X_n(\omega) = \frac{\omega n}{n+1}$, $\omega \in
  \Omega = \mathbb{R}$. Ispitati sigurnu konvergenciju.
\end{exmp}

Vrijedi
\begin{equation}
  \lim_{n\to\infty} X_n(\omega) = \omega \lim_{n\to\infty}\frac{n}{n+1} = \omega
\end{equation}

Dakle, sekvenca $X_n(\omega)$ konvergira svuda ka slučajnoj varijabli $X(\omega)
= \omega$. U ovom primjeru slučajna sekvenca $X_n$ konvergira za svako $\omega$.
Međutim granična vrijednost zavisi od $\omega$, tj. od ishoda eksperimenta, što
je opisano slučajnom varijablom $X$.

\begin{exmp} \label{ex:converge-surely}
  Neka je $X_n(\omega) = \omega^n$, pri čemu je $\Omega= \mathbb{R}$ eksperiment
  opisan uniformnom PDF:
  $$p(\omega) = 1_{[0,1]}(\omega) = \begin{cases}
      1, \omega \in [0,1]\ \\ 0,\ \text{\normalfont inače}
    \end{cases}
  $$
  Ispitati sigurnu konvergenciju sekvence $X_n$.
\end{exmp}
Jasno je da sekvenca $\omega^n$ konvergira ka $X=0$ za $\omega \in [0,1)$, te ka
$X=1$ za $\omega = 1$. Međutim, za sve ostale vrijednosti $\omega$, sekvenca
divergira. Dakle, sekvenca $X_n$ ne konvergira svuda.\\

Međutim, konvergencija svuda u praksi često nije potrebna. Na primjer, nije od
velikog interesa da li sekvenca $X_n$ iz prethodnog primjera konvergirati za
$\omega \in \mathbb{R}\setminus[0,1]$, jer je taj događaj nemoguć. Zato se uvodi
pojam konvergencije skoro svuda.

\begin{definition}[Konvergencija skoro svuda]
  Za slučajnu sekvencu $\{X_n\}$ se kaže da konvergira skoro svuda (skoro
  sigurno, ili sa vjerovatnoćom 1) ka slučajnoj varijabli $X$ ako je
  \begin{equation}
    \lim_{n\to\infty} X_n(\omega) = X(\omega)
  \end{equation}
  za svako $\omega$ iz nekog skupa $\Lambda$ za koji vrijedi
  $\pr(\omega\in\Lambda)=1$.
\end{definition}

Sekvenca iz primjera \ref{ex:converge-surely} konvergira skoro sigurno jer
sekvenca konvergira za $\forall \omega \in \Lambda = [0,1]$, pri čemu je
$\pr(\omega \in [0,1]) = 1$.

\begin{definition}[Konvergencija po vjerovatnoći]
  Za slučajnu sekvencu $\{X_n\}$ se kaže da konvergira po vjerovatnoći ka
  slučajnoj varijabli $X$ ako za svako fiksno $\varepsilon > 0$ vrijedi
  \footnote{
    Ovaj uslov se često (ekvivalentno) navodi kao
    $\lim_{n\to\infty} \pr(|X_n-X|>\varepsilon) = 0$
  }
  \begin{equation}
    \lim_{n\to\infty} \pr\left(|X_n - X| \le \varepsilon\right) = 1
  \end{equation}
\end{definition}

U nastavku će sa $P_X(x)$ biti označena funkcija raspodjele
vjerovatnoće slučajne varijable $X$.
\begin{definition}[Konvergencija po distribuciji]
  Za slučajnu sekvencu $\{X_n\}$ se kaže da konvergira po raspodjeli ka
  slučajnoj varijabli $X$ ako je
  \begin{equation}
    \lim_{n\to\infty} P_{X_n}(x) = P_X(x)
  \end{equation}
  u svakoj tački $x$ u kojoj je funkcija $P_X$ neprekidna.
\end{definition}

TODTOTOTOTOTOTOTOTOTOTODDDDDDDDDDO

\begin{equation} \label{eq:sample-mean}
  S_n = \frac{1}{n} \sum_{i=1}^{n} X_i
\end{equation}

\subsection{Generalizacije i srodne tvdnje}

